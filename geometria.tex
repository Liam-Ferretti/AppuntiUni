\documentclass[a4paper,12pt]{article}

% pacchetti comuni
% -------------------------------
% Pacchetti di base
% -------------------------------
\usepackage[utf8]{inputenc}
\usepackage[T1]{fontenc}
\usepackage[italian]{babel}
\usepackage{amsmath, amssymb, amsthm, mathtools, physics}
\usepackage{graphicx}
\usepackage{tikz}
\usepackage{hyperref}
\usepackage{geometry}
\geometry{a4paper, margin=2.5cm}

% -------------------------------
% Per codice (Lab Calcolo)
% -------------------------------
\usepackage{listings}
\usepackage{xcolor}

\lstnewenvironment{code}
{\lstset{
		language=C++,
		inputencoding=utf8,
		extendedchars=true,
		literate={à}{{\`a}}1
		{è}{{\`e}}1
		{é}{{\'e}}1
		{ì}{{\`i}}1
		{ò}{{\`o}}1
		{ù}{{\`u}}1,
		basicstyle=\ttfamily\footnotesize,
		numbers=left,
		numberstyle=\tiny,
		frame=single,
		breaklines=true,
		showstringspaces=false,
		commentstyle=\color{gray},  % commenti verdi
		keywordstyle=\color{cyan},      % parole chiave azzurre
		stringstyle=\color{orange}      % stringhe arancioni
}}
{}





\usetikzlibrary{arrows.meta}
\usetikzlibrary{angles,quotes, arrows.meta}

\usepackage{verbatim}

\newcommand{\rel}[1][R]{R}
\newcommand{\Mod}[1]{\ (\mathrm{mod}\ #1)}


% info documento
\title{Appunti di Geometria}
\author{Liam Ferretti}

\date{\today}

\begin{document}
	
	\maketitle
	
	\begin{abstract}
		Le informazioni sul corso si trovano sul \hyperlink{https://sites.google.com/uniroma1.it/kieranogrady}{sito del docente}.
		
		Di regola il lunedì verranno svolti esercizi o chiariti dubbi, e le lezioni saranno svolte da S. Molcho.
		
		Ogni settimana (probabilmente il giovedì) verranno caricati degli esercizi su classroom da riconsegnare entro domenica sera.
		
		Il ricevimento avrà luogo nello studio 137 nell'edificio CU006 il martedì dalle 11:15 alle 12:45.
		
		Le dispense sono disponibili sul sito, il libro non è necessario.
	\end{abstract}
		
	\newpage
	\tableofcontents
	\clearpage
	
	\section{Insiemi}
	Per insieme si intende una collezione di oggetti, detti elementi.
	Preso l'insieme $X$ e $a$ un elemento, allora:
	\[
	a \in X: \text{ significa che "a è un elemento di X"}
	\]
	
	\[
	a \not \in X: \text{ significa che "a NON è un elemento di X"}
	\]
	
	Per definire un insieme si usa questa notazione:
	\[
	X := \{a | a \text{ ha la proprietà P}\}
	\]
	
	Es: 
	\[
		X_a := \{a \in \mathbb{N} \mid 2 | a\} = \{0, 2, 4, 6, 8, ...\}
	\]
	Con $2 \mid a$ si intende che 2 è un divisore di a, quindi che a è pari.\newline
	
	Esiste un insieme chiamato insieme vuoto che non contiene nessun elemento ed è rappresentato con: $\varnothing$ \newline
	
	È possibile dichiarare una famiglia di insiemi numerati da un altro insieme in questo modo:
	\[
	\{X_i\}_{i \in I}
	\]
	
	Es: 
	\[
	X_a := \{m \in \mathbb{Z} \mid a | m\}
	\]
	Allora:
	\begin{align*}
		X_0 &:= \{0\} \\
		X_1 &:= \mathbb{Z} \\
		X_2 &:= \{0, \pm 2, \pm 4, ...\}
	\end{align*}
	\newline	
	Insiemi che è necessario conoscere:
	\[
		\mathbb{N} = \{ \text{numeri naturali} \}
	\]
	\[
		\mathbb{Z} = \{ \text{numeri interi} \}
	\]
	\[
		\mathbb{Q} = \{ \text{numeri razionali} \} = \left\{ \frac{p}{q} \;\middle|\; p,q \in \mathbb{Z}, \; q \neq 0 \right\}
	\]
	\[
		\mathbb{R} = \{ \text{numeri reali} \}
	\]
	\[
		\mathbb{C} = \{ \text{numeri complessi} \}
	\]
	
	\subsection{Sotto insieme}
	Presi due insiemi $X, Y$, $X$ è sotto insieme di $Y$, se ogni elemento di X è elemento di Y, formalmente si esprime con:
	\[
	X \subset Y \iff \forall x \in X, x \in Y
	\]
	
	OSS: X è sotto insieme di se stessa in quanto contiene tutti i suoi elementi, quindi ha senso dire che:
	\[
	X \subset X
	\]
	
	\subsection{Operazioni tra insiemi}
	Le operazioni che si possono effettuare tra insiemi sono diverse:
	\begin{itemize}
		\item Unione, rappresentata da $\cup$
		\item Intersezione, rappresentata da $\cap$
		\item Prodotto cartesiano, rappresentata da $\times$
	\end{itemize}
	
	\subsubsection{Unione}
	Preso l'insieme:
	 
	\[
	X_i := \{m \in \mathbb{Z} \mid i | m\}
	\]
	L'unione degli insiemi $X_i$ è l'insieme $X \mid x\in X \iff \exists i \in I \mid x \in X_i$
	
	\[
	X = \bigcup_{i \in I} X_i
	\]
	
	Se $I = \{1, 2\} \rightarrow X_1 \cup X_2$
	
	Se $I = \{1, 2, 3\} \rightarrow X_1 \cup X_2 \cup X_3$
		
	Se $\displaystyle I = \mathbb{Z} \rightarrow \bigcup_{i \in I} X_i = \mathbb{Z}$
	
	\subsubsection{Intersezione}
	Preso l'insieme:
	
	\[
	X_i := \{m \in \mathbb{Z} \mid i | m\}
	\]
	L'intersezione degli insiemi $X_i$ è l'insieme $X \mid x\in X \iff \forall i \in I \exists x \in X_i$
	
	\[
	X = \bigcap_{i \in I} X_i
	\]
	
	Se $I = \{1, 2\} \rightarrow X_1 \cap X_2$
	
	Se $I = \{1, 2, 3\} \rightarrow X_1 \cap X_2 \cap X_3$
	
	Se $\displaystyle I = \mathbb{Z} \rightarrow \bigcap_{i \in I} X_i = \{0\}$	
	
	\subsubsection{Prodotto cartesiano}
	Il prodotto cartesiano di due insiemi $X, Y$, è definito come l'insieme i cui elementi sono le coppie ordinate $(x,y)$, con $x\in X, y\in Y$.
	\[
	X = Y = \mathbb{R} \rightarrow \mathbb{R} \times \mathbb{R} := \{(x, y) \mid x,y\in \mathbb{R}\}
	\]
	\[
	\mathbb{R} \times \mathbb{R} \rightarrow \mathbb{R}^2
	\]
	Se si hanno più insiemi:
	\[
	X_1 \times X_2 \times X_3 \times ... \times X_n := \{(x_1, x_2, x_3, ... ,x_n) \mid x_n \in X_n¨\}
	\]
	
	\section{Applicazione tra insiemi}
	
	Presi gli insiemi $X, Y$, si definisce un'applicazione $f$ da $X$ (insieme di input o \textbf{dominio}) a $Y$ (insieme di output o \textbf{codominio}) come una legge che associa a ogni elemento $x \in X$ un elemento $f(x) \in Y$. La notazione è:
	\[
	X \xlongrightarrow{f} Y
	\]
	ovvero, l'applicazione manda l'insieme $X$ nell'insieme $Y$.  
	Se si considerano i singoli elementi degli insiemi, si scrive:
	\[
	x \mapsto f(x)
	\]

	Es:
	\begin{align*}
		\mathbb{Z} & \xlongrightarrow{f} \mathbb{Z} \\
		m & \longmapsto 3m
	\end{align*}
	
	Affinché 2 applicazioni sono uguali devono coincidere: \textbf{dominio}, \textbf{codominio} e \textbf{funzione}.
	
	\subsection{Composizione di applicazioni}
	Presi gli insiemi $X$, $Y$, $Z$ e le applicazioni $f$ e $g$ allora:
	
	\[
	\begin{tikzpicture}[>=stealth, node distance=2cm]
		\node (X) {$X$};
		\node (Y) [right of=X] {$Y$};
		\node (Z) [right of=Y] {$Z$};
		
		\draw[->] (X) -- node[above] {$g$} (Y);
		\draw[->] (Y) -- node[above] {$f$} (Z);
		
		\node (x) [below of=X] {$x$};
		\node (gx) [below of=Y] {$g(x)$};
		\node (fgx) [below of=Z] {$f(g(x))$};
		
		\draw[|-{Stealth}] (x) -- (gx);
		\draw[|-{Stealth}] (gx) -- (fgx);
		
	\end{tikzpicture}
	\]
	è possibile definire la composizione di g e f, ovvero una applicazione del tipo:
	
	\[
	X \xrightarrow{f \circ g} Z
	\]
	
	$f \circ g$, si legge f composto g, ed è definito come:
	\[
	f \circ g := f(g(x)) \forall x\in X
	\]
	
	Es: avendo tre insiemi $\mathbb{Z}$, e due applicazioni $f$ e $g$, allora:
	\[
	\begin{array}{c c c c c}
		X & \xlongrightarrow{g} & Y & \xlongrightarrow{f} & Z \\
		n & \longmapsto & 3n+1 & & \\
		&& n & \longmapsto & n^2
	\end{array}
	\]
	
	Allora le composizioni di applicazioni sono:
	
	\[
	(f \circ g)(n) := f(3n + 1) = 9m^2 + 6m + 1
	\]	
	
	\[
	(g \circ f)(n) := g(n^2) = 3n^2 + 1
	\]
	Bisogna notare che in questo caso ha senso sia $f \circ g$ sia $g \circ f$, ma $(f \circ g) \not = (g \circ f)$
	
	
	OSS: 
	\[
	X \xlongrightarrow{g} Y \xlongrightarrow{f} X
	\]
	In questo caso e solo nel caso in cui l'insieme iniziale è lo stesso di quello finale, hanno senso sia $f \circ g$ sia $g \circ f$:
	\begin{center}
		Per $f \circ g : X \xlongrightarrow{f \circ g} X$ e per $g \circ f : Y \xlongrightarrow{g \circ f} Y$
	\end{center}
	Se $f \circ g = g \circ f$, allora $X = Y$, ma se $X = Y$ non è certo che $f \circ g = g \circ f$
	
	\subsection{Proprietà associativa della composizione}
	
	Avendo 4 insiemi $X, Y, W, Z$ e applicazioni $f, g, h$
	\[
	X \xlongrightarrow{f} Y \xlongrightarrow{g} W \xlongrightarrow{h} Z
	\]
	allora vale
	\[
	h \circ (g \circ f) = (h \circ g) \circ f
	\]
	Dimostrazione:
	\begin{align}
		(h \circ (g \circ f))(x) = h(g \circ f(x)) = h(g(f(x))) \\
		((h \circ g) \circ f))(x) = (h \circ g)(f(x)) = h(g(f(x)))
	\end{align}
	
	Perciò $\forall x \in X: (h \circ (g \circ f))(x) = ((h \circ g) \circ f))(x)$, quindi la composizione di applicazioni è una proprietà associativa, in cui il modo in cui si raggruppano le parentesi non cambia il risultato finale
	
	\subsection{Insieme identità di una applicazione}
	L'insieme identità di $X$ è l'applicazione:
	\[
	\begin{array}{c c c}
		X & \xlongrightarrow{Id_x} & X \\
		x & \longmapsto & x \\
	\end{array}
	\]
	Può essere espressa sia come $Id_x$ sia come $1_x$
	
	\setcounter{equation}{0}
	\begin{align}
		X & \xlongrightarrow{1_X} X \xlongrightarrow{f} Y \\
		X & \xlongrightarrow{f} Y \xlongrightarrow{1_Y} Y
	\end{align}
	Nel caso (1) l'applicazione composta $(f \circ 1_x) = f$ e nel caso (2) l'applicazione $(1_x \circ f) = f$.
	
	\subsection{Iniettività}
	Presi due insiemi $X, Y$ e una applicazione $f$:
	\[
	X \xlongrightarrow{f} Y
	\] 
	
	\begin{center}
		$f$ è iniettiva $\iff \forall x_1,x_2 \in X \rightarrow f(x_1) = f(x_2) \Rightarrow x_1 = x_2$.
	\end{center}
	
	Se presi due elementi $x_1, x_2 \in X$ allora gli elementi del codominio sono diversi se e solo se $x_1 \not = x_2$
	
	Es: 
	\[
	\begin{array}{c c c}
		\mathbb{R} & \xlongrightarrow{f} & \mathbb{R} \\
		x & \longmapsto & 3x + 1\\
	\end{array}
	\]
	è iniettiva in quanto:
	\[
	f(x_1) = f(x_2) \iff 3x_1 + 1 = 3x_2 + 1 \iff 3(x_1 - x_2) = 0 \iff x_1 = x_2
	\]
	
	\subsection{Suriettività}
	Presi due insiemi $X, Y$ e una applicazione $f$:
	\[
	X \xlongrightarrow{f} Y
	\] 
	
	\begin{center}
		$f$ è suriettiva $\iff \forall y \in Y \; \exists x\in X \mid f(x) = y$.
	\end{center}
	
	L'immagine di $f$, ovvero, Im $f$ è definito come: 
	\[
	\text{Im} f:= \{y \in Y \mid \exists x\in X \mid f(x) = y\}
	\]
	OSS: f è suriettiva $\iff$ Im$f = Y$
	\subsection{Biettività}
	Presi due insiemi $X, Y$ e una applicazione $f$:
	\[
	X \xlongrightarrow{f} Y
	\] 
	
	\[
	f \text{ è biunivoca} \iff \forall y \in Y \; \exists! x \in X \text{ tale che } f(x) = y 
	\]
	\[
	\text{(con } \exists! \text{ si intende esiste ed è unico})
	\]
	
	
	\subsection{Applicazione inversa}
	Presi due insiemi $X, Y$ e una applicazione $f$ biunivoca:
	\[
	X \xlongrightarrow{f} Y
	\] 
	
	L'applicazione inversa di $f$ è l'applicazione:
	\[
	\begin{array}{c c c}
		Y & \xlongrightarrow{f^{-1}} & X \\
		y & \longmapsto & x \in X \mid \exists ! x \mid f^{-1}(x) = y
	\end{array}
	\]
	
	Es:
	scelto
	\[
	\begin{array}{c c c}
		\mathbb{R} & \xlongrightarrow{f} & \mathbb{R} \\
		x & \longmapsto & 3x + 1
	\end{array}
	\]
	è biunivoca in quanto è sia suriettiva sia iniettiva, perciò è possibile trovare $f^{-1}$ risolvendo per x:
	\[
	3x + 1 = y \Longrightarrow 3x = y - 1 \Longrightarrow x = \dfrac{y -1}{3}
	\]
	Quindi l'applicazione inversa è:
	\[
	\begin{array}{c c c}
		\mathbb{R} & \xlongrightarrow{f^{-1}} & \mathbb{R} \\
		y & \longmapsto & \dfrac{y -1}{3}
	\end{array}
	\]
	
	È quindi possibile vedere che l'applicazione composta tra $f$ e $f^{-1}$ in qualsiasi ordine rappresenta l'applicazione identità:	
	\[
	\begin{array}{c c c c c c}
		X & \xlongrightarrow{f} & Y & \xlongrightarrow{f^{-1}} & X: & f^{-1} \circ f = Id_x \\
		Y & \xlongrightarrow{f^{-1}} & X & \xlongrightarrow{f} & Y: & f \circ f^{-1} = Id_x
	\end{array}
	\]\newline
	Presi due insiemi $X, Y$ e l'applicazione $f$:
	\[
	X \xlongrightarrow{f} Y
 	\]
 	
	se $y \in Y \rightarrow f^{-1}(y) := \{x \in X | f(x) = y\}$, in questo caso ha senso anche se non si tratta di applicazioni biunivoche in quanto restituisce un insieme e non un singolo elemento, quindi nel caso esistano più $x | f(x) = y$ si otterrà come risultato l'insieme numerico che contiene tutte le $x$
	
	\section{Relazioni}
	Preso $X$ un insieme, una relazione su X è un sottoinsieme $\rel \subset X^2 = X \times X$ 
	
	Per $x \rel y$ si intende che $(x, y)\in \rel$
	
	Es 1: 
	\[
	X = \{\text{cittadini italiani}\}
	\]
	\[
	\rel = \{(x, y) \mid \text{x e y sono coetanei}\}
	\]
	\[
	x\rel y \rightarrow \text{x e y sono coetanei}
	\]
	Es 2:
	
	$X = \mathbb{Z} \text{, scelto } n \in \mathbb{Z}$
	\[
	\rel_n = \{(x, y) \mid n|(x - y), x,y \in X\} = x \equiv y \Mod{n}, \text{questa è la relazione di congruenza modulo n}
	\]
	
	\subsection{Relazioni di equivalenza}
	Per relazione di equivalenza si intende una relazione $\rel$ su $X$, con $X$ un insieme qualsiasi, in cui valgono 3 proprietà:
	\begin{itemize}
		\item Riflessiva: $x \rel x, \text{ con } x \in X$
		\item Simmetrica: $x \rel y \rightarrow y \rel x, \text{ con } x,y \in X$
		\item Transitiva: $x \rel y \wedge y \rel z \rightarrow x \rel z, \text{ con } x,y,z \in X$
	\end{itemize}
	Controllo se l´esempio 2 è una relazione di equivalenza:
	\[
	X = \mathbb{Z}, n\in \mathbb{Z}
	\]
	\[
	x \rel y \text{ se } x \equiv y \Mod n
	\]
	\begin{itemize}
		\item Riflessiva: $x \equiv x \Mod n$ se $n|(x - x) \rightarrow n|0$, vera
		\item Simmetrica: $x \equiv y \Mod n$ se $n|(x - y) \rightarrow x-y = n*q \rightarrow -x + y = n * (-q)$, vera
		\item Transitiva: $x \equiv y \Mod n \wedge y \equiv z \Mod n \rightarrow x \equiv z \Mod{n}$, dimostrazione:
		\[
		n|(x-y) \wedge n|(y - x) = (x - y = q * n) \wedge (y - z = r * n)
		\]
		sommando due numeri multipli di n ottengo un multiplo di n
		\[
		(x -y) + (y -z) = q * n + r * n = n (q+r)
		\]
		quindi è verificata la proprietà transitiva
	\end{itemize}
	
	\subsubsection{Classe di equivalenza}
	Sia $x \in X$, allora la classe di equivalenza di x è:
	\[
	[x] := \{y \in X \mid x \rel y\}
	\]
	fissato x, $[x]$ è l'insieme composta dagli elementi $y$ $t.c.$ $x \rel y$
	
	\subsubsection{Quoziente di X modulo R o insieme quoziente}
	Definiti $X$ un insieme e $\rel$ una relazione, il quoziente di X modulo $\rel$ è l'insieme i cui elementi sono le classi R-equivalenza, ovvero le classi di equivalenza rispetto alla relazione $\rel$ su $X$, quindi l'insieme quoziente è l'insieme composto dalle classi di equivalenza in $X$ generate da $\rel$ ed è rappresentato da: $X/R$
	
	\[
	X/R := \{[x] \tc x \in X\}
	\]
	
	In riferimento all'esempio 1:
	\[
	C/R := \{[0], [1], [2], ... , \text{ [età massima] }\} %% check notazione
	\], e ogni elemento ad esempio [19], ha la proprio classe di equivalenza:
	\[
	[19] = \{x \in C \mid eta(x) = 19\}
	\]
	
	\newpage
	
	\section{Anello e campo}
	A è un insieme i cui elementi sono coppie ordinate $(x,y)$, ed ha 2 operazioni:
	\begin{itemize}
		\item somma: 
		\[
		\begin{array}{c c c}
			A \times A & \xlongrightarrow{somma} & A \\
			(x, y) & \longmapsto & x + y \\
		\end{array}
		\]
		\item prodotto:
		\[
		\begin{array}{c c c}
			A \times A & \xlongrightarrow{prodotto} & A \\
			(x, y) & \longmapsto & x * y \\
		\end{array}
		\]		
	\end{itemize}
	Le proprietà che rendono un insieme un anello o un campo sono:
	\begin{enumerate}
		\item Esistenza ed unicità dell'elemento neutro della somma, quindi:
		\[\exists! 0\in A \tc 0 + z = z + 0 = z \forall z\in A\]
		dimostrazione: presi $a \in A, b,c \in A \tc a + b = 0 \wedge a + c = 0, b \not = c$
		
		ipotesi: $b \stackrel{?}{=} c$
		\[0 = 0 \rightarrow a + b = a + c \rightarrow b = c\]
		\[Q.E.D\]
		\item Proprietà commutativa della somma, quindi: 
		\[z_1 + z_2 = z_2 + z_1, \forall z_1,z_2\in A\]
		\item Proprietà associativa della somma, quindi:
		\[z_1 + (z_2+z_3) = (z_1 + z_2) + z_3, \forall z_1,z_2,z_3\in A\]
		\item Esistenza ed unicità dell'opposto di $z \in A$, quindi:
		\[\forall z \in A \exists! w\in A \tc z + w = 0\]
		\item Proprietà associativa del prodotto, quindi:
		\[(z_1 * z_2) * z_3 = z_1 * (z_2 * z_3), \forall z_1,z_2,z_3\in A\]
		\item Proprietà distributiva del prodotto rispetto alla somma, quindi:
		\[z_1 * (z_2 + z_3) = z_1 * z_2 + z_1 * z_3 \wedge (z_1 + z_2) * z_3 = z_1 * z+3 + z_2 * z_3, \forall z_1,z_2,z_3\in A\]
		\item Proprietà commutativa rispetto al prodotto, quindi:
		\[w * z = z * w, \forall w,z \in A\]
		\item L'esistenza ed unicità dell'unità neutra del prodotto ($u \in A$) (unità moltiplicativa), diversa dall'unità neutra della somma, quindi $\not = 0$:
		\[u * x = x * u = x, \forall x \in A\]
		\item L'esistenza ed unicità dell'inverso moltiplicativo, quindi:
		\[\forall x\not = 0 \in A, \exists! w\in A \tc x * w = 1 \rightarrow w = x^{-1}\]
	\end{enumerate}
	
	\subsection{Anello}
	A (con le operazioni definite) è un anello se valgono le prime 6 proprietà (o assiomi).
	
	D'ora in avanti per completezza e facilità in diverse situazioni, per anello si intende un anello commutativo rispetto al prodotto e che presenta l'unità moltiplicativa, quindi in cui valgono anche le proprietà 7 e 8.
	
	\subsection{Campo}
	A (con le operazioni definite) è un campo se valgono le prime 9 proprietà (o assiomi).
		
	\section{Numeri complessi}
	Definite due operazioni in $\mathbb{R}^2$:
	\begin{itemize}
		\item Somma:
		\[(x_1, y_1), (x_2, y_2) \in \mathbb{R}^2\]
		\[(x_1, y_1) + (x_2, y_2) := (x_1 + x_2, y_1 + y_2)\]
		\item Prodotto:
		\[(x_1, y_1), (x_2, y_2) \in \mathbb{R}^2\]
		\[(x_1, y_1) * (x_2, y_2) := ((x_1 * x_2 - y_1*y_2), (x_1 * y_2 + x_2 * y_1))\]
	\end{itemize}
	il prodotto è così perchè se fosse definito come$(x_1 * x_2 , y_1 * y_2)$ allora $\mathbb{R}^2$ non sarebbe un anello (commutativo ed unitario)
	
	\subsection{Verifica $\mathbb{R}^2$}
	Verifichiamo che $\mathbb{R}^2$ con le proprietà definite è un campo:
	\begin{enumerate}
		\item Unicità dell'elemento neutro della somma:
		\[(0, 0) \not = (0^\prime, 0^\prime)\]
		\[(0, 0) + (x, y) = (x,y) \wedge (0^\prime, 0^\prime) + (x, y) = (x, y)\]
		\[(x, y) = (x, y) \rightarrow (0, 0) = (0^\prime, 0^\prime)\]
		\item Proprietà commutativa della somma:
		\[(a, b) + (c, d) = (a + c, b + d) = (c + d) + (a + b)\]
		la proprietà commutativa degli elementi $a,b,c,d$ non va dimostrata in questo momento essendo essi elementi di $\mathbb{R}$ commutativi.
		\item Associatività della somma:
		\[(a, b) + ((c, d) + (e, f)) = (a, b) + (c + e, d + f) = (c + e + a, d + f+ b)\]
		\[((a,b) +  (c, d)) + (e, f) = (a + c, b + d) + (e, f) = (a + c + e, b + d + f)\]
		\[(c + e + a, d + f+ b) = (a + c + e, b + d + f)\]
		per la proprietà commutativa in $\mathbb{R}$
		\item Esistenza ed unicità dell'opposto:
		\[(z, w) \not = (z^\prime, w^\prime)\]
		\[(x, y) + (z, w) = (0, 0) \wedge (x, y) + (z^\prime, w^\prime) = (0, 0)\]
		\[(0, 0) = (0, 0) \rightarrow (x, y) + (z, w) = (x, y) + (z^\prime, w^\prime)\]
		\[(z, w) = (z^\prime, w^\prime)\]
		quindi esiste ed è unico l'opposto della somma
	\end{enumerate}
	
	Oss: 
	\[(1, 0) * (x, y) = (1 * x - 0 * y, 1 * y + 0 * x) = (x, y)\]
	quindi la coppia $(1, 0)$ è l'elemento neutro del prodotto
	
	\subsection{Notazione}
	\[
	\left\{
	\begin{aligned}
		1:= (1, 0) \quad & i := (0, 1) \\
		\forall a \in \mathbb{R} \quad \quad & a(x, y) :=(ax, ay) \\
		i^2 = -1 \quad \;\;\;& 
	\end{aligned}
	\right.
	\]
	Un numero complesso $z = a1 + bi = a + bi := (a, 0) + (0, b) = (a, b)$
	
	Allora abbiamo che un prodotto tra numeri complessi è in questa forma:
	\[(a + bi) * (c + di) = a * c - bi * c + a * di + b * d * i^2 = (a*c - b*d + (a*d + b*c)i)\]
	
	Il prodotto in C, tra coppie, deriva:
	\[z = a + bi = (a, b) \wedge w = c + di = (c, d)\] 
	\[z * w = (a, b) * (c, d) = (a + bi) * (c + di)\]
	A questo punto svolgo le normali moltiplicazioni tra parentesi in R, perché non sono più coppie, ma sono numeri,  da questo ottengo una somma tra una parte reale e una parte immaginaria, da cui deriva il prodotto in $\mathbb{C}$.
	\[(a*c - b*d + (a*d + b*c)i) = ((a*c - b*d), (a*d + b*c))\]
	
	\subsection{Continuo verifica $\mathbb{R}$}
	continuo con la verifica delle proprietà dalla 5 alla 9
	\begin{enumerate}
		\setcounter{enumi}{4} % parte da 3 perché LaTeX aggiunge +1
		\item Associatività del prodotto:
		\[((a, b) * (c, d)) * (x, y) = (a * c - b * d, a*d + b*c) * (x, y)\]
		\[(A, B) := (a * c - b * d, a*d + b*c) \rightarrow (A, B) * (x, y)\]
		\[(A * x - B*y, A*y + B*x) = ((a * c - b * d) * x - (a*d + b*c)*y, (a * c - b * d) * y + (a*d + b*c)*x)\]
		\[((a * c * x - b*d*x - a*d*y -b*c*y), (a*c*y - b*d*y + a*d*x+b*c*x))\]
		ora vediamo se la proprietà associativa vale:
		\[(a,b) * ((c, d) * (x, y)) = (a,b) * (c*x - d*y, c*y + d*x)\]		
		\[(C, D) := (c*x - d*y, c*y + d*x) \rightarrow (a,b) * (C, D)\]
		\[(a* C - b*D, a*D + b*C) = (a*(c*x - d*y) - b*(c*y + d*x), a*(c*y + d*x) + b*(c*x - d*y))\]
		\[((a*c*x - a*d*y - b*c*y - b*d*x), (a*c*y + a*d*x + b*c*x - b*d*y))\]
		
		secondo la proprietà commutativa della somma, i due prodotti sono equivalenti, quindi la proprietà associativa è verificata
			
		\item Proprietà associativa del prodotto rispetto alla somma:
		\[z_1 * (z_2 + z_3) = (a,b) * ((c,d) + (x,y))\]		
		\[(c,d) + (x,y) = (c+x, d+y) \rightarrow (a,b) * (c+x, d+y)\]		
		\[(a,b)*(c+x, d+y) = (a*(c+x) - b*(d+y), a*(d+y) + b*(c+x))\]		
		\[(a*c + a*x - b*d - b*y, a*d + a*y + b*c + b*x)\]		
		\[(a*c - b*d, a*d + b*c) + (a*x - b*y, a*y + b*x)\]			
		\[(a,b)*(c,d) + (a,b)*(x,y)\]		
		\[\Rightarrow z_1*(z_2 + z_3) = z_1*z_2 + z_1*z_3\]
		
		\item unità neutra del prodotto
				
		\item Esistenza ed unicità dell'inverso moltiplicativo:
		\[(a, b) * (x, y) = (1, 0) = 1 + 0i\]
		\[(ax - by, ay + bx)\]
		Dato che vogliamo che la parte reale del prodotto sia uguale a uno e quella immaginaria sia zero, possiamo scrivere questa equazione come un sistema.
		\[
		\left\{
		\begin{aligned}
			ax - by & = 1\\
			ay + bx & = 0
		\end{aligned}
		\right.
		\]
		Dal secondo:
		\[ay + bx = 0 \implies y = -\frac{b}{a}x, a \not = 0\]
		Sostituendo nella prima equazione:
		\[ax - b(-\frac{b}{a}x) = 1 \implies ax + \frac{b^2}{a}x = 1 \implies a^2x+b^2x = a \implies x = \frac{a}{a^2 + b^2}\]
		Sapendo che $y = -\frac{b}{a}x$, allora:
		\[y = -\frac{b}{a} \frac{a}{a^2 + b^2} = \frac{b}{a^2 + b^2}\]
		Quindi abbiamo dimostrato che la coppia $\left(\dfrac{a}{a^2 + b^2}, \dfrac{b}{a^2 + b^2}\right) = (a, b)^{-1}$ quindi:
		\[(a, b) * \left(\dfrac{a}{a^2 + b^2}, \dfrac{b}{a^2 + b^2}\right) = (1, 0)\] 
	\end{enumerate}
	
	\subsection{Caratteristiche}
	Un numero $z \in \mathbb{C} = a + bi, a,b \in \mathbb{R}$, e con $\mathbb{R}$ in intende $\mathbb{R}^2$ con le operazioni di somma e prodotto definite.
	La parte reale dei numeri complessi è definita da $Re(z) := a$ e la parte immaginaria $Im(z) := b$
	
	\subsection{Rappresentazione grafica}
	Scelto un sistema di coordinate cartesiano del piano, allora ad $a + bi \in C$, corrisponde un punto/vettore $P(a,b)$. Presi $z = a+bi, w = c+ di$
	\begin{center}
		\begin{tikzpicture}[->, thick]
			% assi
			\draw[->] (-3,0) -- (3,0) node[above] {$Re$};
			\draw[->] (0,-3) -- (0,3) node[above] {$Im$};
			
			\fill (1, 0) circle (2pt) node[below] {$(1, 0)$};
			\fill (0, 1) circle (2pt) node[left] {$(0, 1)$};
			
			% vettore
			\draw[->, red] (0,0) -- (0.5,1) node[above] {$z$};
			\draw[->, blue] (0,0) -- (2,0.5) node[above] {$w$};
			
			
			\draw[->, red, dashed] (2,0.5) -- (2.5, 1.5) node[above] {};
			
			\draw[->, darkgray] (0, 0) -- (2.5, 1.5) node[above] {$w + z$};
		\end{tikzpicture}
	\end{center}
	la somma tra due numeri $w,z \in \mathbb{C}$ equivale alla somma tra vettori. 
	
	Per il prodotto tra $z_1, z_2 \in \mathbb{C}$ bisogna definire il modulo di numero complesso:
	\[|z| = \rho := \sqrt{a^2 + b^2}\]
	e l'argomento di z:
	\[Arg(z) = \theta, \text{ determinato a meno di sommare un multiplo intero di 2$\pi$}, z\not = 0\]
	
	\begin{center}
		\begin{tikzpicture}[->, thick]
			    % assi
			\draw[->] (-3,0) -- (3,0) node[above] {$Re$};
			\draw[->] (0,-3) -- (0,3) node[above] {$Im$};
			
			% punti base
			\fill (1,0) circle (2pt) node[below] {$(1,0)$};
			\fill (0,1) circle (2pt) node[left] {$(0,1)$};
			
			% vettore z
			\draw[->, black, thick] (0,0) -- (2,0.8) node[above] {$z$};
			
			\draw[black] (0,0) -- (2,0.8) node[midway, above] {$\rho$};
			
			% nodi
			\coordinate (X) at (2,0);
			\coordinate (O) at (0,0);
			\coordinate (Z) at (2,0.8);
			
			% angolo senza freccia
			\pic [draw=black, "$\theta$", angle eccentricity=1.5, angle radius=1cm, mark=none]
			{angle = X--O--Z};
		\end{tikzpicture}
	\end{center}
	
	sapendo che le coordinate di z son $(a, b)$, allora posso definirle in funzione di $\rho, \theta$:
	\[
	\left\{
	\begin{aligned}
		a =& \rho * \cos\theta\\
		b =& \rho * \sin\theta
	\end{aligned}
	\right. \Rightarrow z = \rho * \cos\theta + i\rho * \sin\theta = \rho( \cos\theta + i\sin\theta)
	\]
	Allora presi: $z_1 = \rho_1( \cos\theta_1 + i \sin\theta_1), z_2 = \rho_2( \cos\theta_2 + i\sin\theta_2)$
	\[ z * w = \rho_1 * \rho_2(\cos\theta_1 + i\sin\theta_1) * (\cos\theta_2 + i\sin\theta_2) = \]
	\[ = \rho_1 * \rho_2(\cos\theta_1 * \cos\theta_2 - \sin\theta_1 * \sin\theta_2) + i(\cos\theta_1 * \sin\theta_2 + \sin\theta_1 * \cos\theta_2 ) = \]
	\[= \rho_1 * \rho_2(\cos(\theta_1+\theta_2) + i\sin(\theta_1+\theta_2))\]
	quindi:
	\[
	\left\{
	\begin{aligned}
		|z_1 * z_2| = |z_1| * |z_1| = \rho_1 * \rho_2 \quad \quad \\
		Arg(z_1 * z_2) = Arg(z_1) + Arg(z_2)
	\end{aligned}\right.\]
	$Arg(z_1 * z_2)$ è determinato a meno di un multiplo intero di 2$\pi$
	
	\begin{center}
		\begin{tikzpicture}[->, thick]
			% assi
			\draw[->] (-3,0) -- (3,0) node[above] {$Re$};
			\draw[->] (0,-3) -- (0,3) node[above] {$Im$};
			
			% punti base
			\fill (1,0) circle (2pt) node[below] {$(1,0)$};
			\fill (0,1) circle (2pt) node[left] {$(0,1)$};
			
			% vettore z_1
			\draw[->, black, thick] (0,0) -- (2,0.8) node[above] {$z_1$};
			
			\draw[black] (0,0) -- (2,0.8) node[midway, above] {$\rho_1$};
			
			% nodi
			\coordinate (X) at (2,0);
			\coordinate (O) at (0,0);
			\coordinate (Z) at (2,0.8);
			
			% angolo senza freccia
			\pic [draw=black, "$\theta_1$", angle eccentricity=1.5, angle radius=1cm, mark=none]
			{angle = X--O--Z};
			
			% vettore z_2
			\draw[->, black, thick] (0,0) -- (2,2.2) node[above] {$z_2$};
			
			\draw[black] (0,0) -- (2,2.2) node[midway, above] {$\rho_2$};
			
			% nodi
			\coordinate (Y) at (2,0);
			\coordinate (O) at (0,0);
			\coordinate (K) at (2,2.2);
			
			% angolo senza freccia
			\pic [draw=black, "$\theta_2$", angle eccentricity=1.3, angle radius=1.5cm, mark=none]			
			{angle = Y--O--K};
			
			% vettore prodotto
    		\draw[->, red, thick] (0,0) -- (69.5:6.1) node[above] {$z_1 * z_2$};
			
		\end{tikzpicture}
	\end{center}	
	
	moltiplicare un numero complesso per $i$, vuol dire ruotarlo di 90 gradi verso sinistra
	
	\subsection{Teorema fondamentale dell'algebra}
	Sia $P(z) = a_0z^n + a_1z^{n-1} + a_2z^{n-2} + ... + a_n$, polinomio in $z$ a coefficienti in $\mathbb{C}$ di grado $n > 0$ con $a_0 \not = 0$, allora:
	\[\exists \xi \in \mathbb{C} \tc P(\xi) = 0\]
	($\xi$ è una "radice di $P(z) = 0$")
	
	T.F.A $\implies \exists \lambda_1, \lambda_2, ..., \lambda_n \in \mathbb{C} \tc P(z) = a_0(z - \lambda_1)(z - \lambda_2) * ... * (z - \lambda_n)$
	
	Per via del teorema di Ruffini in quanto:
	\begin{center}
		essendo $\lambda_1$ radice di $P(z) = 0 \rightarrow P(z) = (z - \lambda_1)*q(z)$, in cui il grado di $q(z)$ = grado di $P(z)$ - 1
	\end{center}
	
	Es:
	
	\[p(z) = z^n - a\]
	si cercano le radici di $z^n - a = 0$:
	\begin{itemize}
		\item $a = 0 \rightarrow z^n = 0$, la radice è unica con molteplicità $n$
		\item $a \not = 0$, se $n = 3$:
		\[
		\begin{aligned}
			z &= \rho(\cos\theta + i\sin\theta) \\
			z^n &= \rho(\cos n\theta + i\sin n\theta) \\
			a \, \, &= \rho_0(\cos\theta_0 + i\sin\theta_0)
		\end{aligned}
		\]
		\[z^n = a \iff \left\{
		\begin{aligned}
			\rho^n &= \rho_0 \rightarrow \rho = \sqrt[n]{\rho_0}\\
			n\theta &= \theta_0 \text{ a meno di multipli interi di } 2\pi
		\end{aligned} \right.\]
		
		\[Arg(z = \sqrt[n]{a}) = \theta = \left\{\frac{\theta_0}{n} + \frac{k2\pi}{n}, k \in \{0, 1, ..., n -1\}\right\}\]
		
	\end{itemize}
	
	
	
\end{document}
