\documentclass[a4paper,12pt]{article}

% pacchetti comuni
% -------------------------------
% Pacchetti di base
% -------------------------------
\usepackage[utf8]{inputenc}
\usepackage[T1]{fontenc}
\usepackage[italian]{babel}
\usepackage{amsmath, amssymb, amsthm, mathtools, physics}

\usepackage{tcolorbox}
\tcbuselibrary{breakable} % per box su più righe

% Comando personalizzato per teoremi/enunciati
\newtcolorbox{teorembox}{
	colback=gray!10,    % sfondo grigio chiaro
	colframe=gray!50,   % bordo grigio
	boxrule=0.5pt,      % spessore bordo
	sharp corners,      % angoli non arrotondati
	breakable,          % consente il ritorno a capo nel box
	left=8pt, right=8pt, top=6pt, bottom=6pt,
}

% Comando semplificato: \teorem{...}
\newcommand{\teorem}[1]{\begin{teorembox}#1\end{teorembox}}
\newcommand{\brackets}[1]{\langle #1 \rangle}
\usepackage{graphicx}
\usepackage{tikz}
\usetikzlibrary{calc}
\usetikzlibrary{intersections}
\usepackage{pgfplots}
\pgfplotsset{compat=1.18}
\usepackage{hyperref}
\usepackage{geometry}
\newtheorem{proposition}{Proposizione}
\geometry{a4paper, margin=2.5cm}
\newcommand{\tc}{\;\text{t.c.}\;}

\usetikzlibrary{arrows.meta}
\usetikzlibrary{angles,quotes, arrows.meta}

\usepackage{verbatim}


% -------------------------------
% Per codice (Lab Calcolo)
% -------------------------------
\usepackage{listings}
\usepackage{xcolor}

\lstnewenvironment{code}
{\lstset{
		language=C,
		inputencoding=utf8,
		extendedchars=true,
		literate={à}{{\`a}}1
		{è}{{\`e}}1
		{é}{{\'e}}1
		{ì}{{\`i}}1
		{ò}{{\`o}}1
		{ù}{{\`u}}1,
		basicstyle=\ttfamily\footnotesize,
		numbers=left,
		numberstyle=\tiny,
		frame=single,
		breaklines=true,
		showstringspaces=false,
		commentstyle=\color{gray},  % commenti verdi
		keywordstyle=\color{cyan},      % parole chiave azzurre
		stringstyle=\color{orange}      % stringhe arancioni
}}
{}

\usepackage{lmodern}
\usepackage{inconsolata} % font monospaziato
\usepackage{xcolor}
\usepackage{listings}

% Stile UNIX
\lstdefinestyle{unixstyle}{
	language=bash,
	basicstyle=\ttfamily\small,
	numbers=left,                     % ✅ numeri di riga
	numberstyle=\tiny\color{gray!70}, % colore numeri
	numbersep=6pt,
	frame=single,
	backgroundcolor=\color{gray!6},
	rulecolor=\color{black!60},
	columns=fullflexible,
	keepspaces=false,                 % ✅ rimuove tab/spazi iniziali
	showstringspaces=false,
	breaklines=true,
	aboveskip=0pt,
	belowskip=0pt,
	xleftmargin=0pt,
	xrightmargin=0pt,
	upquote=true,
	literate={\$}{{\textcolor{black}{\$}}}1
}

% Definizione del comando \unix{...}
\newcommand{\unix}[1]{%
	\lstinline[style=unixstyle]!#1!%
}




\usetikzlibrary{arrows.meta}

% info documento
\title{Appunti di Geometria}
\author{Liam Ferretti}

\date{\today}

\begin{document}
	
	\maketitle
	
	\begin{abstract}
		Le informazioni sul corso si trovano sul \hyperlink{https://sites.google.com/uniroma1.it/kieranogrady}{sito del docente}.
		
		Di regola il lunedì verranno svolti esercizi o chiariti dubbi, e le lezioni saranno svolte da S. Molcho.
		
		Ogni settimana (probabilmente il giovedì) verranno caricati degli esercizi su classroom da riconsegnare entro domenica sera.
		
		Il ricevimento avrà luogo nello studio 137 nell'edificio CU006 il martedì dalle 11:15 alle 12:45.
		
		Le dispense sono disponibili sul sito, il libro non è necessario.
	\end{abstract}
		
	\tableofcontents
	\clearpage
	
	\section{Insiemi}
	Per insieme si intende una collezione di oggetti, detti elementi.
	Preso l'insieme $X$ e $a$ un elemento, allora:
	\[
	a \in X: \text{ significa che "a è un elemento di X"}
	\]
	
	\[
	a \not \in X: \text{ significa che "a NON è un elemento di X"}
	\]
	
	Per definire un insieme si usa questa notazione:
	\[
	X := \{a | a \text{ ha la proprietà P}\}
	\]
	
	Es: 
	\[
		X_a := \{a \in \mathbb{N} \mid 2 | a\} = \{0, 2, 4, 6, 8, ...\}
	\]
	Con $2 \mid a$ si intende che 2 è un divisore di a, quindi che a è pari.\newline
	
	Esiste un insieme chiamato insieme vuoto che non contiene nessun elemento ed è rappresentato con: $\varnothing$ \newline
	
	È possibile dichiarare una famiglia di insiemi numerati da un altro insieme in questo modo:
	\[
	\{X_i\}_{i \in I}
	\]
	
	Es: 
	\[
	X_a := \{m \in \mathbb{Z} \mid a | m\}
	\]
	Allora:
	\begin{align*}
		X_0 &:= \{0\} \\
		X_1 &:= \mathbb{Z} \\
		X_2 &:= \{0, \pm 2, \pm 4, ...\}
	\end{align*}
	\newline	
	Insiemi che è necessario conoscere:
	\[
		\mathbb{N} = \{ \text{numeri naturali} \}
	\]
	\[
		\mathbb{Z} = \{ \text{numeri interi} \}
	\]
	\[
		\mathbb{Q} = \{ \text{numeri razionali} \} = \left\{ \frac{p}{q} \;\middle|\; p,q \in \mathbb{Z}, \; q \neq 0 \right\}
	\]
	\[
		\mathbb{R} = \{ \text{numeri reali} \}
	\]
	\[
		\mathbb{C} = \{ \text{numeri complessi} \}
	\]
	
	\subsection{Sotto insieme}
	Presi due insiemi $X, Y$, $X$ è sotto insieme di $Y$, se ogni elemento di X è elemento di Y, formalmente si esprime con:
	\[
	X \subset Y \iff \forall x \in X, x \in Y
	\]
	
	OSS: X è sotto insieme di se stessa in quanto contiene tutti i suoi elementi, quindi ha senso dire che:
	\[
	X \subset X
	\]
	
	\subsection{Operazioni tra insiemi}
	Le operazioni che si possono effettuare tra insiemi sono 2:
	\begin{itemize}
		\item Unione, rappresentata da $\cup$
		\item Intersezione, rappresentata da $\cap$
	\end{itemize}
	
	\subsubsection{Unione}
	Preso l'insieme:
	 
	\[
	X_i := \{m \in \mathbb{Z} \mid i | m\}
	\]
	L'unione degli insiemi $X_i$ è l'insieme $X \mid x\in X \iff \exists i \in I \mid x \in X_i$
	
	\[
	X = \bigcup_{i \in I} X_i
	\]
	
	Se $I = \{1, 2\} \rightarrow X_1 \cup X_2$
	
	Se $I = \{1, 2, 3\} \rightarrow X_1 \cup X_2 \cup X_3$
		
	Se $\displaystyle I = \mathbb{Z} \rightarrow \bigcup_{i \in I} X_i = \mathbb{Z}$
	
	\subsubsection{Intersezione}
	Preso l'insieme:
	
	\[
	X_i := \{m \in \mathbb{Z} \mid i | m\}
	\]
	L'intersezione degli insiemi $X_i$ è l'insieme $X \mid x\in X \iff \forall i \in I \exists x \in X_i$
	
	\[
	X = \bigcap_{i \in I} X_i
	\]
	
	Se $I = \{1, 2\} \rightarrow X_1 \cap X_2$
	
	Se $I = \{1, 2, 3\} \rightarrow X_1 \cap X_2 \cap X_3$
	
	Se $\displaystyle I = \mathbb{Z} \rightarrow \bigcap_{i \in I} X_i = \{0\}$	
	
	\section{Applicazione tra insiemi}
	
	Presi gli insiemi $X, Y$, si definisce un'applicazione $f$ da $X$ (insieme di input o \textbf{dominio}) a $Y$ (insieme di output o \textbf{codominio}) come una legge che associa a ogni elemento $x \in X$ un elemento $f(x) \in Y$. La notazione è:
	\[
	X \xlongrightarrow{f} Y
	\]
	ovvero, l'applicazione manda l'insieme $X$ nell'insieme $Y$.  
	Se si considerano i singoli elementi degli insiemi, si scrive:
	\[
	x \mapsto f(x)
	\]

	Es:
	\begin{align*}
		\mathbb{Z} & \xlongrightarrow{f} \mathbb{Z} \\
		m & \longmapsto 3m
	\end{align*}
	
	Affinché 2 applicazioni sono uguali devono coincidere: \textbf{dominio}, \textbf{codominio} e \textbf{funzione}.
	
	\subsection{Composizione di applicazioni}
	Presi gli insiemi $X$, $Y$, $Z$ e le applicazioni $f$ e $g$ allora:
	
	\[
	\begin{tikzpicture}[>=stealth, node distance=2cm]
		\node (X) {$X$};
		\node (Y) [right of=X] {$Y$};
		\node (Z) [right of=Y] {$Z$};
		
		\draw[->] (X) -- node[above] {$g$} (Y);
		\draw[->] (Y) -- node[above] {$f$} (Z);
		
		\node (x) [below of=X] {$x$};
		\node (gx) [below of=Y] {$g(x)$};
		\node (fgx) [below of=Z] {$f(g(x))$};
		
		\draw[|-{Stealth}] (x) -- (gx);
		\draw[|-{Stealth}] (gx) -- (fgx);
		
	\end{tikzpicture}
	\]
	è possibile definire la composizione di g e f, ovvero una applicazione del tipo:
	
	\[
	X \xrightarrow{f \circ g} Z
	\]
	
	$f \circ g$, si legge f composto g, ed è definito come:
	\[
	f \circ g := f(g(x)) \forall x\in X
	\]
	
	Es: avendo tre insiemi $\mathbb{Z}$, e due applicazioni $f$ e $g$, allora:
	\[
	\begin{array}{c c c c c}
		X & \xlongrightarrow{g} & Y & \xlongrightarrow{f} & Z \\
		n & \longmapsto & 3n+1 & & \\
		&& n & \longmapsto & n^2
	\end{array}
	\]
	
	Allora le composizioni di applicazioni sono:
	
	\[
	(f \circ g)(n) := f(3n + 1) = 9m^2 + 6m + 1
	\]	
	
	\[
	(g \circ f)(n) := g(n^2) = 3n^2 + 1
	\]
	Bisogna notare che in questo caso ha senso sia $f \circ g$ sia $g \circ f$, ma $(f \circ g) \not = (g \circ f)$
	
	
	OSS: 
	\[
	X \xlongrightarrow{g} Y \xlongrightarrow{f} X
	\]
	In questo caso e solo nel caso in cui l'insieme iniziale è lo stesso di quello finale, hanno senso sia $f \circ g$ sia $g \circ f$:
	\begin{center}
		Per $f \circ g : X \xlongrightarrow{f \circ g} X$ e per $g \circ f : Y \xlongrightarrow{g \circ f} Y$
	\end{center}
	Se $f \circ g = g \circ f$, allora $X = Y$, ma se $X = Y$ non è certo che $f \circ g = g \circ f$
	
	\subsection{Proprietà associativa della composizione}
	
	Avendo 4 insiemi $X, Y, W, Z$ e applicazioni $f, g, h$
	\[
	X \xlongrightarrow{f} Y \xlongrightarrow{g} W \xlongrightarrow{h} Z
	\]
	allora vale
	\[
	h \circ (g \circ f) = (h \circ g) \circ f
	\]
	Dimostrazione:
	\begin{align}
		(h \circ (g \circ f))(x) = h(g \circ f(x)) = h(g(f(x))) \\
		((h \circ g) \circ f))(x) = (h \circ g)(f(x)) = h(g(f(x)))
	\end{align}
	
	Perciò $\forall x \in X: (h \circ (g \circ f))(x) = ((h \circ g) \circ f))(x)$, quindi la composizione di applicazioni è una proprietà associativa, in cui il modo in cui si raggruppano le parentesi non cambia il risultato finale
	
	\subsection{Insieme identità di una applicazione}
	L'insieme identità di $X$ è l'applicazione:
	\[
	\begin{array}{c c c}
		X & \xlongrightarrow{Id_x} & X \\
		x & \longmapsto & x \\
	\end{array}
	\]
	Può essere espressa sia come $Id_x$ sia come $1_x$
	
	\setcounter{equation}{0}
	\begin{align}
		X & \xlongrightarrow{1_X} X \xlongrightarrow{f} Y \\
		X & \xlongrightarrow{f} Y \xlongrightarrow{1_Y} Y
	\end{align}
	Nel caso (1) l'applicazione composta $(f \circ 1_x) = f$ e nel caso (2) l'applicazione $(1_x \circ f) = f$.
	
	\subsection{Iniettività}
	Presi due insiemi $X, Y$ e una applicazione $f$:
	\[
	X \xlongrightarrow{f} Y
	\] 
	
	\begin{center}
		$f$ è iniettiva $\iff \forall x_1,x_2 \in X \rightarrow f(x_1) = f(x_2) \Rightarrow x_1 = x_2$.
	\end{center}
	
	Se presi due elementi $x_1, x_2 \in X$ allora gli elementi del codominio sono diversi se e solo se $x_1 \not = x_2$
	
	Es: 
	\[
	\begin{array}{c c c}
		\mathbb{R} & \xlongrightarrow{f} & \mathbb{R} \\
		x & \longmapsto & 3x + 1\\
	\end{array}
	\]
	è iniettiva in quanto:
	\[
	f(x_1) = f(x_2) \iff 3x_1 + 1 = 3x_2 + 1 \iff 3(x_1 - x_2) = 0 \iff x_1 = x_2
	\]
	
	\subsection{Suriettività}
	Presi due insiemi $X, Y$ e una applicazione $f$:
	\[
	X \xlongrightarrow{f} Y
	\] 
	
	\begin{center}
		$f$ è suriettiva $\iff \forall y \in Y \; \exists x\in X \mid f(x) = y$.
	\end{center}
	
	L'immagine di $f$, ovvero, Im $f$ è definito come: 
	\[
	\text{Im} f:= \{y \in Y \mid \exists x\in X \mid f(x) = y\}
	\]
	OSS: f è suriettiva $\iff$ Im$f = Y$
	\subsection{Biettività}
	Presi due insiemi $X, Y$ e una applicazione $f$:
	\[
	X \xlongrightarrow{f} Y
	\] 
	
	\[
	f \text{ è biunivoca} \iff \forall y \in Y \; \exists! x \in X \text{ tale che } f(x) = y 
	\]
	\[
	\text{(con } \exists! \text{ si intende esiste ed è unico})
	\]
	
	
	\subsection{Applicazione inversa}
	Presi due insiemi $X, Y$ e una applicazione $f$ biunivoca:
	\[
	X \xlongrightarrow{f} Y
	\] 
	
	L'applicazione inversa di $f$ è l'applicazione:
	\[
	\begin{array}{c c c}
		Y & \xlongrightarrow{f^{-1}} & X \\
		y & \longmapsto & x \in X \mid \exists ! x \mid f^{-1}(x) = y
	\end{array}
	\]
	
	Es:
	
	\[
	\begin{array}{c c c}
		\mathbb{R} & \xlongrightarrow{f} & \mathbb{R} \\
		x & \longmapsto & 3x + 1
	\end{array}
	\]
	è biunivoca in quanto è sia suriettiva sia iniettiva, perciò è possibile trovare $f^{-1}$ risolvendo per x:
	\[
	3x + 1 = y \Longrightarrow 3x = y - 1 \Longrightarrow x = \dfrac{y -1}{3}
	\]
	Quindi l'applicazione inversa è:
	\[
	\begin{array}{c c c}
		\mathbb{R} & \xlongrightarrow{f^{-1}} & \mathbb{R} \\
		y & \longmapsto & \dfrac{y -1}{3}
	\end{array}
	\]
	
	È quindi possibile vedere che l'applicazione composta tra $f$ e $f^{-1}$ in qualsiasi ordine rappresenta l'applicazione identità:	
	\[
	\begin{array}{c c c c c c}
		X & \xlongrightarrow{f} & Y & \xlongrightarrow{f^{-1}} & X: & f^{-1} \circ f = Id_x \\
		Y & \xlongrightarrow{f^{-1}} & X & \xlongrightarrow{f} & Y: & f \circ f^{-1} = Id_x
	\end{array}
	\]\newline
	Presi due insiemi $X, Y$ e l'applicazione $f$:
	\[
	X \xlongrightarrow{f} Y
 	\]
 	
	se $y \in Y \rightarrow f^{-1}(y) := \{x \in X | f(x) = y\}$, in questo caso ha senso anche se non si tratta di applicazioni biunivoche in quanto restituisce un insieme e non un singolo elemento, quindi nel caso esistano più $x | f(x) = y$ si otterrà come risultato l'insieme numerico che contiene tutte le $x$
	
\end{document}
