\documentclass[a4paper,12pt]{article}

% pacchetti comuni
% -------------------------------
% Pacchetti di base
% -------------------------------
\usepackage[utf8]{inputenc}
\usepackage[T1]{fontenc}
\usepackage[italian]{babel}
\usepackage{amsmath, amssymb, amsthm, mathtools, physics}

\usepackage{tcolorbox}
\tcbuselibrary{breakable} % per box su più righe

% Comando personalizzato per teoremi/enunciati
\newtcolorbox{teorembox}{
	colback=gray!10,    % sfondo grigio chiaro
	colframe=gray!50,   % bordo grigio
	boxrule=0.5pt,      % spessore bordo
	sharp corners,      % angoli non arrotondati
	breakable,          % consente il ritorno a capo nel box
	left=8pt, right=8pt, top=6pt, bottom=6pt,
}

% Comando semplificato: \teorem{...}
\newcommand{\teorem}[1]{\begin{teorembox}#1\end{teorembox}}
\newcommand{\brackets}[1]{\langle #1 \rangle}
\usepackage{graphicx}
\usepackage{tikz}
\usetikzlibrary{calc}
\usetikzlibrary{intersections}
\usepackage{pgfplots}
\pgfplotsset{compat=1.18}
\usepackage{hyperref}
\usepackage{geometry}
\newtheorem{proposition}{Proposizione}
\geometry{a4paper, margin=2.5cm}
\newcommand{\tc}{\;\text{t.c.}\;}

\usetikzlibrary{arrows.meta}
\usetikzlibrary{angles,quotes, arrows.meta}

\usepackage{verbatim}


% -------------------------------
% Per codice (Lab Calcolo)
% -------------------------------
\usepackage{listings}
\usepackage{xcolor}

\lstnewenvironment{code}
{\lstset{
		language=C,
		inputencoding=utf8,
		extendedchars=true,
		literate={à}{{\`a}}1
		{è}{{\`e}}1
		{é}{{\'e}}1
		{ì}{{\`i}}1
		{ò}{{\`o}}1
		{ù}{{\`u}}1,
		basicstyle=\ttfamily\footnotesize,
		numbers=left,
		numberstyle=\tiny,
		frame=single,
		breaklines=true,
		showstringspaces=false,
		commentstyle=\color{gray},  % commenti verdi
		keywordstyle=\color{cyan},      % parole chiave azzurre
		stringstyle=\color{orange}      % stringhe arancioni
}}
{}

\usepackage{lmodern}
\usepackage{inconsolata} % font monospaziato
\usepackage{xcolor}
\usepackage{listings}

% Stile UNIX
\lstdefinestyle{unixstyle}{
	language=bash,
	basicstyle=\ttfamily\small,
	numbers=left,                     % ✅ numeri di riga
	numberstyle=\tiny\color{gray!70}, % colore numeri
	numbersep=6pt,
	frame=single,
	backgroundcolor=\color{gray!6},
	rulecolor=\color{black!60},
	columns=fullflexible,
	keepspaces=false,                 % ✅ rimuove tab/spazi iniziali
	showstringspaces=false,
	breaklines=true,
	aboveskip=0pt,
	belowskip=0pt,
	xleftmargin=0pt,
	xrightmargin=0pt,
	upquote=true,
	literate={\$}{{\textcolor{black}{\$}}}1
}

% Definizione del comando \unix{...}
\newcommand{\unix}[1]{%
	\lstinline[style=unixstyle]!#1!%
}




\usetikzlibrary{backgrounds}
\usepackage{parcolumns}

% info documento
\title{Appunti di Analisi Matematica}
\author{Liam Ferretti}
\date{\today}

\begin{document}
	
	\maketitle
	
	\begin{abstract}
		Per i ricevimenti bisogna prenotarsi via e-mail, e si svolgeranno nell'edificio 105 dell'edificio Castel Nuovo
		Si potranno trovare note ed esercizi su e-learning
		
		Programma:
		\begin{itemize}
		\item Numeri reali  
		\item Funzioni di variabili reali  
		\item Successioni e serie
		\item Limiti e continuità
		\item Calcolo differenziale ad una variabile  
		\item Integrali
		\item Equazioni differenze lineari  
		\item Funzioni di più variabili (tutti i capitoli precedenti comprendono le funzioni a più variabili)
		\end{itemize}
		I libri di testo sono presenti su e-learning, ed è consigliato "Crasta Malusa", da cui assegnerà gli esercizi.
		
		L'esame sarà composto da scritto più orale, non ci sarà probabilmente un esonero, e con l'orale si può incrementare o decrementare il voto di fino a 3 punti in positivo o 3 in negativo, tranne nel caso in cui si commettano errori su: limiti, continuità o divisione per 0, che comporta la bocciatura immediata.
	\end{abstract}
	
	\newpage
	\tableofcontents
	\clearpage
	
	\section{Cenni di logica ed insiemistica}
		La matematica si costruisce su:
		\begin{itemize}
			\item elementi di base:
				\subitem oggetti di base (enti primitivi)
				\subitem proprietà di basa (assiomi)
			\item regole di deduzione che sono fissate
		\end{itemize}
	\subsection{Notazione}
	La notazione si divide in:
	\begin{itemize}
		\item Connettiva:
			\subitem $\neg$ , non
			\subitem $\vee$ ,  e
			\subitem $\wedge$ , o
			\subitem $\Rightarrow$ , implica 
			\subitem $\iff$ , equivale (se e solo se)
			\subitem : (t.c.) , tale che / tale per cui
		\item Quantificativa:
			\subitem $\exists$ , esiste
			\subitem $\not \exists$ , non esiste
			\subitem $\exists!$ , esiste ed è unico
			\subitem $\forall$ , per ogni
	\end{itemize}
	\subsection{Le proposizioni}
	Per proposizione Si intende una affermazione.
	
	Es:
		\begin{itemize}
			\item P = oggi è martedì
			\item $\neg$P = oggi non è martedì
			\item Q = c'è il sole
			\item P $\wedge$ Q = oggi è martedì e c'è il sole
			\item P $\vee$ Q = oggi è martedì oppure c'è il sole
			
			\item $\neg$(P $\wedge$ Q) $\Rightarrow$ P $\vee$ Q può essere vera (P oppure Q), ma P $\wedge$ Q non può essere vera (P e Q), quindi non possono essere vere allo stesso tempo
		\end{itemize}
	A $\Rightarrow$ B, vuol dire se A è vero allora B è vero. \newline
	A $\iff$ B = (A $\Rightarrow$ B) $\wedge$ (B $\Rightarrow$ A), e vuol dire se e solo se A allora B.\newline
	Partendo dalla proposizione precedente, è vero che:
	\[
	A \Rightarrow B \iff \neg B \Rightarrow \neg A
	\]
	cioè è utile nelle dimostrazioni per assurdo. Nelle dimostrazioni si parte dagli assiomi e con le regole logiche si fanno ipotesi (affermazioni) che nel caso in cui fosse vera rende la tesi (la validità di una o più proprietà).
	
	OSS: è \textbf{sbagliato} dire che:
	\[
	A \Rightarrow B \iff \neg A \Rightarrow \neg B
	\]
	in quanto il non avvenire di A non implica che B non possa avvenire per altre motivazioni.
	
	\subsection{Insiemi}
	Un insieme è una collezioni di elementi
	\subsubsection{Notazione degli insiemi}
	\begin{itemize}
		\item definizione di insieme: G $:= \{e_1, e_2, e_3, ...\}$, è necessario l'uso di $:=$ per definire un insieme, e vuol dire "definito come".
		\item quando due insiemi hanno gli stessi elementi si dichiara l'uguaglianza tra $I_1$ e $I_2$, con il simbolo =, ad esempio
		\[
		F := \{0, 1\}, H := \{1, 0\} \rightarrow F = H
		\]
		\item per definire l'appartenenza di un elemento in un insieme si scrive $a \in I$,  se questo elemento non appartiene all'insieme si rappresenta $a \not \in I$ con $a$ un elemento qualsiasi e $I$ un insieme qualsiasi.
		
	\end{itemize}
	\subsubsection{Relazione di ordine o inclusione}
	Se $A_1 \subset A_2 \rightarrow A_1$ è contenuto in $A_2$, e $A_1$ è al tal più grande quanto $A_2$, ovvero $A_1$ è un sotto insieme di $A_2$.
	
	Se i due insiemi non sono uguali allora si segna $A_1 \not \subseteq A_2$, è quindi strettamente contenuto.
	
	Se invece i due insiemi possono essere uguali, si scrive $A_1 \subseteq A_2$.
	\newpage
	Es:
	\[
	\begin{tikzpicture}
		% Insieme grande X
		\draw (0,0) ellipse (3cm and 2cm);
		\node at (3.5,0) {$A_2$}; % scritta attaccata al bordo destro
		
		% Sottoinsieme Y
		\draw (0,0) ellipse (1.5cm and 1cm);
		\node at (2,0) {$A_1$}; % scritta vicino al bordo destro
		
		\node at (-0.5,0.3) {$a$};
		\node at (0.4,-0.2) {$b$};
		\node at (0.5,+0.5) {$c$};
	\end{tikzpicture}
	\] 
	contengono gli stessi elementi quindi: $A_1 = A_2$ 
	\[
	\begin{tikzpicture}
		% Insieme grande X
		\draw (0,0) ellipse (3cm and 2cm);
		\node at (3.5,0) {$A_2$}; % scritta attaccata al bordo destro
		
		% Sottoinsieme Y
		\draw (0,0) ellipse (1.5cm and 1cm);
		\node at (2,0) {$A_1$}; % scritta vicino al bordo destro
		
		\node at (-0.5,0.3) {$a$};
		\node at (0.4,-0.2) {$b$};
		\node at (0.5,+0.5) {$c$};
		\node at (-2,-1) {$d$};
		\node at (-1.8, -0.2) {$e$};
	\end{tikzpicture}
	\]
	in questo caso $A_2$ contiene più elementi di $A_1$, quindi $A_1 \not \subseteq A_2$
	
	\subsubsection{Proprietà degli insiemi}
	Gli insiemi hanno 3 proprietà principali:
	\begin{itemize}
		\item Riflessiva: $A \subseteq A$, per $A$ insieme qualsiasi, quindi l'insieme contiene se stesso
		\item Antisimmetrica: $(A \subseteq B) \wedge (B \subseteq A) \Rightarrow A = B$, per $A, B$ insiemi qualsiasi
		\item Transitiva:$(A \subseteq B) \wedge (B \subseteq C) \Rightarrow A \subseteq C$, per $A, B, C$ insiemi qualsiasi
	\end{itemize}
	
	\subsubsection{Operazioni tra insiemi}	
	Presi due insiemi $A, B$ allora esistono diverse proprietà:
	\begin{itemize}
		\item Unione (o): $A \cup B$ := $\{a : a\in A \vee a \in B\}$
		\[
		\begin{tikzpicture}
			
			\begin{scope}
				\clip (0,0) circle (1.5cm);
				\fill[blue!30] (-2,-2) rectangle (4,2);
			\end{scope}
			
			\begin{scope}
				\clip (2,0) circle (1.5cm);
				\fill[blue!30] (-2,-2) rectangle (4,2);
			\end{scope}
			
			\draw (0,0) circle (1.5cm);
			\node at (-2,0) {$A$};
			\draw (2,0) circle (1.5cm);
			\node at (4,0) {$B$};
		\end{tikzpicture}
		\]
		\item Intersezione (e): $A \cap B$ := $\{a : a\in A \wedge a \in B\}$ 
		\[
		\begin{tikzpicture}
			\begin{scope}
				\clip (0,0) circle (1.5cm);
				\fill[blue!30] (2,0) circle (1.5cm);
			\end{scope}
			
			\draw (0,0) circle (1.5cm);
			\node at (-2,0) {$A$};
			\draw (2,0) circle (1.5cm);
			\node at (4,0) {$B$};		
		\end{tikzpicture}
		\]
		\item Differenza (-): $A \setminus B $ :=  $\{a : a \in A, a\not \in B\}$
		\[
		\begin{tikzpicture}
			% Coloro A \ B
			\begin{scope}
				\clip (0,0) circle (1.5cm);
				\fill[blue!30] (-2,-2) rectangle (2,2);
			\end{scope}
			
			% Rimuovo l’intersezione con B
			\begin{scope}
				\clip (2,0) circle (1.5cm);
				\fill[white] (-2,-2) rectangle (4,2);
			\end{scope}
			
			
			\draw (0,0) circle (1.5cm);
			\node at (-2,0) {$A$};
			\draw (2,0) circle (1.5cm);
			\node at (4,0) {$B$};	
		\end{tikzpicture}
		\]
		\item Prodotto cartesiano: $A \times B$ := $\{(a, b) : a \in A \wedge b \in B\}$
		\[
		\begin{tikzpicture}
			% A e B come insiemi puntuali
			\foreach \x in {1/2,2/2,3/2,4/2,5/2,6/2} {
				\foreach \y in {1/2,2/2,3/2,4/2,5/2,6/2} {
					\fill[blue!30] (\x,\y) circle (0.1cm);
				}
			}
			
			% Etichette
			\node[left] at (0,2) {$B$};
			\node[below] at (2,0) {$A$};
		\end{tikzpicture}
		\]
	\end{itemize}
	
	\subsection{Predicato}
	Una preposizione può dipendere da una o più variabili, ovvero un ente che varia in un gruppo, in quel caso prende il nome di predicato. 
	Es: 
	
	$P$ = oggi è martedì
	
	$P(x)$ = $x$ è martedì 
	
	allora preso $A := \{lunedì, martedì, ... , domenica\}, x \in A$
	
	$B := \{x\in A : P(x)\}$ = $\{martedì\}$, con $P(x)$ si intendono le x che rendono $P(x)$ vera, quindi si cercano le x appartenenti ad $A$ t.c. $P(x)$ sia vera.
	
	\subsubsection{Confronto simbologia logica e insiemistica}
	La simbologia nella logica e nella insiemistica è diversa, ma i termini sono gli stessi: \newline
	\begin{center}
		\begin{tabular}{p{0.45\textwidth} p{0.45\textwidth}}
			\textbf{Logica} &
			\textbf{Insiemistica} \\
			A, ovvero A è vera &
			$A := \{x \in I : A(x)\}$, con $A \subset I$ \\
			$\neg A$, ovvero A non è vera &
			$A^c := \{x \in B : \neg A(x)\}$, con $A^c \not\subset B$ \\
			$\neg (A \wedge B) = \neg A \vee \neg B$ &
			$(A \cap B)^c = A^c \cup B^c$ \\
			$\neg (A \vee B) = \neg A \wedge \neg B$ &
			$(A \cup B)^c = A^c \cap B^c$ \\
			$\Rightarrow$, ad esempio $A \Rightarrow B$ &
			$A \subset B$, perchè A è definito come gli elementi x appartenenti ad un insieme $I$ t.c. $A(x)$ sia vera, allo stesso tempo B è definito come gli elementi x appartenenti ad un insieme $I$ t.c. $B(x)$ sia vera, perciò dire che A implica B, vuol dire che gli elementi x che rendono veri A sono contenuti in B \\
			$\iff$, ad esempio $A \iff B$ & $A = B$, riprendendo la stessa argomentazione in questo caso B è vera se A è vera, ma allo stesso tempo A è vera se B è vera, perciò i due insiemi coincideranno
		\end{tabular}
	\end{center}
	
	\subsubsection{Notazione}
	\begin{itemize}
		\item $x\in A \xRightarrow{def}$ x è elemento di A 
		\item $x\not \in A \xRightarrow{def}$ x non è elemento di A, quindi $x\in A^c$
		\item $A \cap B := \{a : A(a) \wedge B(a)\} = \{x : x \in A \wedge x \in B\} \}$
		\item $A \cup B := \{a : A(a) \vee B(a)\} = \{x : x \in A \vee x \in B\}$
		\item $A \setminus B := \{a : A(a) \vee \neg B(a)\} = \{x : x\in A \vee x \not \in B\} = \{x : x\in A \vee x \in B^c\}$
		\item $A \Delta B := (A \setminus B ) \cup (B \setminus A) = (A \cap B)^c$
	\end{itemize}
	
	\subsubsection{Insieme parti di I}
	L'insieme parti di I, è definito come:
	\[
	P(I) := \{X : X \subset I\}, \text{con $X$ insieme}
	\]
	Preso $I := \{0, 1\} \Rightarrow P(I) = \{0, 1, \{0, 1\}\}$, $P(I)$ rappresenta l'insieme parti, ovvero l'insieme composto da tutti i possibili sottoinsiemi di $I$
	
\end{document}
