\documentclass[a4paper,12pt]{article}

% pacchetti comuni
% -------------------------------
% Pacchetti di base
% -------------------------------
\usepackage[utf8]{inputenc}
\usepackage[T1]{fontenc}
\usepackage[italian]{babel}
\usepackage{amsmath, amssymb, amsthm, mathtools, physics}
\usepackage{graphicx}
\usepackage{tikz}
\usepackage{hyperref}
\usepackage{geometry}
\geometry{a4paper, margin=2.5cm}

% -------------------------------
% Per codice (Lab Calcolo)
% -------------------------------
\usepackage{listings}
\usepackage{xcolor}

\lstnewenvironment{code}
{\lstset{
		language=C++,
		inputencoding=utf8,
		extendedchars=true,
		literate={à}{{\`a}}1
		{è}{{\`e}}1
		{é}{{\'e}}1
		{ì}{{\`i}}1
		{ò}{{\`o}}1
		{ù}{{\`u}}1,
		basicstyle=\ttfamily\footnotesize,
		numbers=left,
		numberstyle=\tiny,
		frame=single,
		breaklines=true,
		showstringspaces=false,
		commentstyle=\color{gray},  % commenti verdi
		keywordstyle=\color{cyan},      % parole chiave azzurre
		stringstyle=\color{orange}      % stringhe arancioni
}}
{}





\usetikzlibrary{backgrounds}
\usepackage{parcolumns}

% info documento
\title{Appunti di Analisi Matematica}
\author{Liam Ferretti}
\date{\today}

\begin{document}
	
	\maketitle
	
	\begin{abstract}
		Per i ricevimenti bisogna prenotarsi via e-mail, e si svolgeranno nell'edificio 105 dell'edificio Castel Nuovo
		Si potranno trovare note ed esercizi su e-learning
		
		Programma:
		\begin{itemize}
		\item Numeri reali  
		\item Funzioni di variabili reali  
		\item Successioni e serie
		\item Limiti e continuità
		\item Calcolo differenziale ad una variabile  
		\item Integrali
		\item Equazioni differenze lineari  
		\item Funzioni di più variabili (tutti i capitoli precedenti comprendono le funzioni a più variabili)
		\end{itemize}
		I libri di testo sono presenti su e-learning, ed è consigliato "Crasta Malusa", da cui assegnerà gli esercizi.
		
		L'esame sarà composto da scritto più orale, non ci sarà probabilmente un esonero, e con l'orale si può incrementare o decrementare il voto di fino a 3 punti in positivo o 3 in negativo, tranne nel caso in cui si commettano errori su: limiti, continuità o divisione per 0, che comporta la bocciatura immediata.
	\end{abstract}
	
	\newpage
	\tableofcontents
	\clearpage
	La matematica si costruisce su:
	\begin{itemize}
		\item elementi di base:
			\subitem oggetti di base (enti primitivi)
			\subitem proprietà di basa (assiomi)
		\item regole di deduzione che sono fissate
	\end{itemize}
	
	\section{Notazione}
	
	La notazione si divide in:
	\begin{itemize}
		\item Connettiva:
			\subitem $\neg$ , non
			\subitem $\vee$ ,  e
			\subitem $\wedge$ , o
			\subitem $\Rightarrow$ , implica 
			\subitem $\iff$ , equivale (se e solo se)
			\subitem : (t.c.) , tale che / tale per cui
		\item Quantificativa:
			\subitem $\exists$ , esiste
			\subitem $\not \exists$ , non esiste
			\subitem $\exists!$ , esiste ed è unico
			\subitem $\forall$ , per ogni
	\end{itemize}
	
	\section{Le proposizioni}
	Per proposizione Si intende una affermazione.
	
	Es:
		\begin{itemize}
			\item P = oggi è martedì
			\item $\neg$P = oggi non è martedì
			\item Q = c'è il sole
			\item P $\wedge$ Q = oggi è martedì e c'è il sole
			\item P $\vee$ Q = oggi è martedì oppure c'è il sole
			
			\item $\neg$(P $\wedge$ Q) $\Rightarrow$ P $\vee$ Q può essere vera (P oppure Q), ma P $\wedge$ Q non può essere vera (P e Q), quindi non possono essere vere allo stesso tempo
		\end{itemize}
	A $\Rightarrow$ B, vuol dire se A è vero allora B è vero. \newline
	A $\iff$ B = (A $\Rightarrow$ B) $\wedge$ (B $\Rightarrow$ A), e vuol dire se e solo se A allora B.\newline
	Partendo dalla proposizione precedente, è vero che:
	\[
	A \Rightarrow B \iff \neg B \Rightarrow \neg A
	\]
	cioè è utile nelle dimostrazioni per assurdo. Nelle dimostrazioni si parte dagli assiomi e con le regole logiche si fanno ipotesi (affermazioni) che nel caso in cui fosse vera rende la tesi (la validità di una o più proprietà).
	
	OSS: è \textbf{sbagliato} dire che:
	\[
	A \Rightarrow B \iff \neg A \Rightarrow \neg B
	\]
	in quanto il non avvenire di A non implica che B non possa avvenire per altre motivazioni.
	
	\section{Insiemi}
	Un insieme è una collezioni di elementi
	\subsection{Notazione degli insiemi}
	\begin{itemize}
		\item definizione di insieme: G $:= \{e_1, e_2, e_3, ...\}$, è necessario l'uso di $:=$ per definire un insieme, e vuol dire "definito come".
		\item quando due insiemi hanno gli stessi elementi si dichiara l'uguaglianza tra $I_1$ e $I_2$, con il simbolo =, ad esempio
		\[
		F := \{0, 1\}, H := \{1, 0\} \rightarrow F = H
		\]
		\item per definire l'appartenenza di un elemento in un insieme si scrive $a \in I$,  se questo elemento non appartiene all'insieme si rappresenta $a \not \in I$ con $a$ un elemento qualsiasi e $I$ un insieme qualsiasi.
		
	\end{itemize}
	\subsection{Relazione di ordine o inclusione}
	Se $A_1 \subset A_2 \rightarrow A_1$ è contenuto in $A_2$, e $A_1$ è al tal più grande quanto $A_2$, ovvero $A_1$ è un sotto insieme di $A_2$.
	
	Se i due insiemi non sono uguali allora si segna $A_1 \not \subseteq A_2$, è quindi strettamente contenuto.
	
	Se invece i due insiemi possono essere uguali, si scrive $A_1 \subseteq A_2$.
	\newpage
	Es:
	\[
	\begin{tikzpicture}
		% Insieme grande X
		\draw (0,0) ellipse (3cm and 2cm);
		\node at (3.5,0) {$A_2$}; % scritta attaccata al bordo destro
		
		% Sottoinsieme Y
		\draw (0,0) ellipse (1.5cm and 1cm);
		\node at (2,0) {$A_1$}; % scritta vicino al bordo destro
		
		\node at (-0.5,0.3) {$a$};
		\node at (0.4,-0.2) {$b$};
		\node at (0.5,+0.5) {$c$};
	\end{tikzpicture}
	\] 
	contengono gli stessi elementi quindi: $A_1 = A_2$ 
	\[
	\begin{tikzpicture}
		% Insieme grande X
		\draw (0,0) ellipse (3cm and 2cm);
		\node at (3.5,0) {$A_2$}; % scritta attaccata al bordo destro
		
		% Sottoinsieme Y
		\draw (0,0) ellipse (1.5cm and 1cm);
		\node at (2,0) {$A_1$}; % scritta vicino al bordo destro
		
		\node at (-0.5,0.3) {$a$};
		\node at (0.4,-0.2) {$b$};
		\node at (0.5,+0.5) {$c$};
		\node at (-2,-1) {$d$};
		\node at (-1.8, -0.2) {$e$};
	\end{tikzpicture}
	\]
	in questo caso $A_2$ contiene più elementi di $A_1$, quindi $A_1 \not \subseteq A_2$
	
	\subsection{Proprietà degli insiemi}
	Gli insiemi hanno 3 proprietà principali:
	\begin{itemize}
		\item Riflessiva: $A \subseteq A$, per $A$ insieme qualsiasi, quindi l'insieme contiene se stesso
		\item Antisimmetrica: $(A \subseteq B) \wedge (B \subseteq A) \Rightarrow A = B$, per $A, B$ insiemi qualsiasi
		\item Transitiva:$(A \subseteq B) \wedge (B \subseteq C) \Rightarrow A \subseteq C$, per $A, B, C$ insiemi qualsiasi
	\end{itemize}
	
	\subsection{Operazioni tra insiemi}	
	Presi due insiemi $A, B$ allora esistono diverse proprietà:
	\begin{itemize}
		\item Unione (o): $A \cup B$ := $\{a : a\in A \vee a \in B\}$
		\[
		\begin{tikzpicture}
			
			\begin{scope}
				\clip (0,0) circle (1.5cm);
				\fill[blue!30] (-2,-2) rectangle (4,2);
			\end{scope}
			
			\begin{scope}
				\clip (2,0) circle (1.5cm);
				\fill[blue!30] (-2,-2) rectangle (4,2);
			\end{scope}
			
			\draw (0,0) circle (1.5cm);
			\node at (-2,0) {$A$};
			\draw (2,0) circle (1.5cm);
			\node at (4,0) {$B$};
		\end{tikzpicture}
		\]
		\item Intersezione (e): $A \cap B$ := $\{a : a\in A \wedge a \in B\}$ 
		\[
		\begin{tikzpicture}
			\begin{scope}
				\clip (0,0) circle (1.5cm);
				\fill[blue!30] (2,0) circle (1.5cm);
			\end{scope}
			
			\draw (0,0) circle (1.5cm);
			\node at (-2,0) {$A$};
			\draw (2,0) circle (1.5cm);
			\node at (4,0) {$B$};		
		\end{tikzpicture}
		\]
		\item Differenza (-): $A \setminus B $ :=  $\{a : a \in A, a\not \in B\}$
		\[
		\begin{tikzpicture}
			% Coloro A \ B
			\begin{scope}
				\clip (0,0) circle (1.5cm);
				\fill[blue!30] (-2,-2) rectangle (2,2);
			\end{scope}
			
			% Rimuovo l’intersezione con B
			\begin{scope}
				\clip (2,0) circle (1.5cm);
				\fill[white] (-2,-2) rectangle (4,2);
			\end{scope}
			
			
			\draw (0,0) circle (1.5cm);
			\node at (-2,0) {$A$};
			\draw (2,0) circle (1.5cm);
			\node at (4,0) {$B$};	
		\end{tikzpicture}
		\]
		\item Prodotto cartesiano: $A \times B$ := $\{(a, b) : a \in A \wedge b \in B\}$
		\[
		\begin{tikzpicture}
			% A e B come insiemi puntuali
			\foreach \x in {1/2,2/2,3/2,4/2,5/2,6/2} {
				\foreach \y in {1/2,2/2,3/2,4/2,5/2,6/2} {
					\fill[blue!30] (\x,\y) circle (0.1cm);
				}
			}
			
			% Etichette
			\node[left] at (0,2) {$B$};
			\node[below] at (2,0) {$A$};
		\end{tikzpicture}
		\]
	\end{itemize}
	
	\section{Predicato}
	Una preposizione può dipendere da una o più variabili, ovvero un ente che varia in un gruppo, in quel caso prende il nome di predicato. 
	Es: 
	
	$P$ = oggi è martedì
	
	$P(x)$ = $x$ è martedì 
	
	allora preso $A := \{lunedì, martedì, ... , domenica\}, x \in A$
	
	$B := \{x\in A : P(x)\}$ = $\{martedì\}$, con $P(x)$ si intendono le x che rendono $P(x)$ vera, quindi si cercano le x appartenenti ad $A$ t.c. $P(x)$ sia vera.
	
	\subsection{Confronto simbologia logica e insiemistica}
	La simbologia nella logica e nella insiemistica è diversa, ma i termini sono gli stessi: \newline
	\begin{center}
		\begin{tabular}{p{0.45\textwidth} p{0.45\textwidth}}
			\textbf{Logica} &
			\textbf{Insiemistica} \\
			A, ovvero A è vera &
			$A := \{x \in I : A(x)\}$, con $A \subset I$ \\
			$\neg A$, ovvero A non è vera &
			$A^c := \{x \in B : \neg A(x)\}$, con $A^c \not\subset B$ \\
			$\neg (A \wedge B) = \neg A \vee \neg B$ &
			$(A \cap B)^c = A^c \cup B^c$ \\
			$\neg (A \vee B) = \neg A \wedge \neg B$ &
			$(A \cup B)^c = A^c \cap B^c$ \\
			$\Rightarrow$, ad esempio $A \Rightarrow B$ &
			$A \subset B$, perchè A è definito come gli elementi x appartenenti ad un insieme $I$ t.c. $A(x)$ sia vera, allo stesso tempo B è definito come gli elementi x appartenenti ad un insieme $I$ t.c. $B(x)$ sia vera, perciò dire che A implica B, vuol dire che gli elementi x che rendono veri A sono contenuti in B \\
			$\iff$, ad esempio $A \iff B$ & $A = B$, riprendendo la stessa argomentazione in questo caso B è vera se A è vera, ma allo stesso tempo A è vera se B è vera, perciò i due insiemi coincideranno
		\end{tabular}
	\end{center}
	
	\subsection{Notazione}
	\begin{itemize}
		\item $x\in A \xRightarrow{def}$ x è elemento di A 
		\item $x\not \in A \xRightarrow{def}$ x non è elemento di A, quindi $x\in A^c$
		\item $A \cap B := \{a : A(a) \wedge B(a)\} = \{x : x \in A \wedge x \in B\} \}$
		\item $A \cup B := \{a : A(a) \vee B(a)\} = \{x : x \in A \vee x \in B\}$
		\item $A \setminus B := \{a : A(a) \vee \neg B(a)\} = \{x : x\in A \vee x \not \in B\} = \{x : x\in A \vee x \in B^c\}$
		\item $A \Delta B := (A \setminus B ) \cup (B \setminus A) = (A \cap B)^c$
	\end{itemize}
	
	\subsection{Insieme parti di I}
	L'insieme parti di I, è definito come:
	\[
	P(I) := \{X : X \subset I\}, \text{con $X$ insieme}
	\]
	Preso $I := \{0, 1\} \Rightarrow P(I) = \{0, 1, \{0, 1\}\}$, $P(I)$ rappresenta l'insieme parti, ovvero l'insieme composto da tutti i possibili sottoinsiemi di $I$.
	Es:
	\[A =\{0, 1\} \Rightarrow P(A) = \{0, 1, \{0, 1\}\}\]
	
	\subsection{Predicato con più variabili}
	\[L(x) = x \text{ segue la lezione}, \forall x \in I\]
	\[P(x, y) = x \text{ segue la lezione il giorno }y, \forall x\in I, y\in G\]
	
	preso $x \in \{\text{studenti del canali 2 del corso Analisi}\} = \{Luca, Liam, ...\}$, allora:
	\begin{itemize}
		\item $\forall x, L(x) \Rightarrow \text{ Luca segue la lezione $\wedge$ Liam segue la lezione $\wedge$ ... }$
		\item $\exists x \tc L(x) \Rightarrow \text{ Luca segue la lezione $\vee$ Liam segue la lezione $\vee$ ... }$
		\item $x = Luca \rightarrow P(Luca, y)$ // Luca segue la lezione il giorno y, se $y = oggi \rightarrow P(Luca, oggi)$ è vero
	\end{itemize}
	Come scrivo che ogni studenti segue la lezione almeno un giorno?
	\[\forall x \in S, \exists y \in G \tc P(x, y)\]
	l'ordine nei quantificatori è importate in quanto dire:
	\[\forall x \in S, \exists y \in G \tc P(x, y)\]
	è diverso da dire:
	\[\exists y \in G, \forall x \in S\tc P(x, y)\]
	che vuol dire "esiste almeno un giorno tale per cui tutti gli studenti vengano a lezione"
	
	\subsection{Negazione del predicato}
	Negare $\forall x \in S, \exists y \in G \tc P(x, y)$ (ogni studenti segue la lezione almeno un giorno), vuol dire:
	\[\neg(\forall x \in S, \exists y \in G \tc P(x, y)) = \exists x \in S, \forall y \in G : \neg P(x, y)\]
	quindi esiste almeno uno studente che non segue mai la lezione
	
	\subsection{Regole della logica}
	\begin{itemize}
		\item se $B \Rightarrow A$ allora A è condizione necessaria per B, quindi:
		\[B \Rightarrow A \rightarrow \neg A \Rightarrow \neg B\]
		\item se $A \Rightarrow B$ allora A è condizione sufficiente di B, quindi basta che A sia vera affinché B avvenga.
		\item se $A \iff B$ allora A è condizione necessaria e sufficente di B.
		\item $\emptyset \in E, \forall E$ insieme
		\item Regola del terzo escluso:
		\[\forall A \text{ insieme}, A \vee \neg A, \text{ quindi succede o non succede}.\]
		\item Principio di non contraddizione:
		\[\forall A, \neg(A \wedge \neg A)\]
		quindi per ogni insieme non è vero che esiste A e non A, in quanto non esisto elementi appartenenti ad A ed anche a non A, quindi non esistono elementi che verificano una proposizione ma allo stesso tempo non la verificano
		\item Transitività:
		\[\forall A, B, C, [(A \Rightarrow B) \wedge (B \Rightarrow C)] \Rightarrow (A \Rightarrow C)\]
	\end{itemize}
	
	\section{Insiemi numerici}
	\subsection{Numeri naturali}
	\[\mathbb{N} := \{0, 1, 2, ..., n\} \text{, ed } \mathbb{N}^+ := \mathbb{N} \setminus \{0\} \]
	in $\mathbb{N}$ è possibili ordinare gli elementi quindi per $m,n \in \mathbb{N}$:
	\[n \leq m \iff \exists p \in \mathbb{N} \tc m = n + p\]
	
	\subsubsection{Proprietà di $\mathbb{N}$ (relazione di ordine)}
	\begin{itemize}
		\item Riflessiva:
		\[\forall n \in \mathbb{N}, n \leq n\]
		\item Antisimmetrica:
		\[\forall n, m \in \mathbb{N}, (n \leq m \wedge m\leq n) \Rightarrow n = m\]
		\item Transitiva:
		\[\forall n, m, p \in \mathbb{N}, (n \leq m \wedge m \leq p) \Rightarrow n \leq p\]
		\item Ordinamento totale:
		\[\forall n, m \in \mathbb{N}, n\leq m \vee m \leq n\]
	\end{itemize}
	
	\subsubsection{Assiomi di Peano}
	Esiste una operazione, \textbf{passaggio al successivo}, $s(n) = n + 1$, tale che:
	\begin{itemize}
		\item (P1) esiste un elemento $0 \in \mathbb{N} \tc 0 \not = s(n), \forall n \in \mathbb{N}$
		\item (P2) se $n,m \in \mathbb{N} \wedge n \not = m \rightarrow s(n) \not = s(m)$
		\item (P3) se $E \subset \mathbb{N}$ è tale che:
		\[ (\text{I1})\quad 0 \in E \quad \quad e \quad \quad (\text{I2})\quad \text{se } n \in E \rightarrow s(n) \in E\]
		allora $E = \mathbb{N}$
	\end{itemize}
	
	\subsubsection{Metodo Induttivo}
	Il P3 è ciò che definisce il metodo induttivo (PI), ovvero:
	\[P(n)\]
	\[E := \{n \in \mathbb{N} \tc P(n)\} \subset \mathbb{N}\]
	PI dice che:
	\[
	\left\{
	\begin{aligned}
		0 &\in E \\
		\forall n &\in E \Rightarrow s(n) \in E
	\end{aligned}\right. \Rightarrow E = \mathbb{N}\]
	
	\begin{proposition}
		$\displaystyle P(n) = \sum_{k=0}^{n} k = \dfrac{n(n+1)}{2}$
	\end{proposition}
	
	\begin{proof}
		Dimostro per induzione
		
		\textbf{Base induttiva:} per \(n = 0\), la sommatoria equivale a \(\frac{0(1)}{2} = 0\)
		
		\textbf{Passo induttivo:} 
		\[P(n + 1) = \sum_{0}^{n + 1} k = \sum_{0}^{n} k + (n + 1) = \frac{n(n+1)}{2} + (n + 1) =\]
		\[= \frac{n(n+1) + 2(n+1)}{2} = \frac{(n+1)(n + 2)}{2} = P(n + 1)\] 
	\end{proof}
	
	Per applicare il metodo induttivo, non è valido partire dalla soluzione per arrivare alla dimostrazione, in quanto bisogna partire da P(n) ed arrivare a dimostrare P(n + 1).
	
	\subsubsection{Principio di buon ordinamento (P.B.O.)}
	Teorema equivalente al principio di induzione, perciò se P1 e P2 sono valide allora PI $\iff P.B.O$, e dice che:
	\[\forall E \subset \mathbb{N} \tc E \not = 0, \exists n_0 \in E \tc \exists n \geq n_0, \forall n \in E\]
	Il minimo di un insieme E, se esiste è definito come:
	\[a\in E \tc a \leq b, \forall b \in E\]
	
	Il principio di buon ordinamento permette di dimostrare induttivamente anche quando la base induttiva è diversa da 0, e in quel caso:
	\[P(b) \wedge P(n) \Rightarrow P(n + 1), \forall n \in \mathbb{N}\setminus \{0, ... ,b\}, b \not = 0\]
	
	\subsubsection{Fattoriale e coefficiente binomiale}
	Due funzioni matematiche che si definiscono per induzione sono:
	\begin{itemize}
		\item Fattoriale:
		\[n! = n(n-1)(n-2)(...)(1), \forall n \in \mathbb{N}\]
		e per definizione il fattoriale di 0 è 1:
		\[0! = 1\]
		\item Coefficiente binomiale:
		\[\binom{n}{k} = \frac{n!}{k!(n-k)!}, \forall n,k \in \mathbb{N}, n \geq k\]
		partendo dal coefficiente binomiale, è possibile svolgere lo sviluppo di una potenza ennesima di un binomio, definito binomio di Newton:
		\[(a+b)^n = \sum_{k = 0}^{n} \binom{n}{k} a^{n-k}b^k, n\in \mathbb{N}^+\]
		in questo binomio il coefficiente binomiale da il coefficiente di $a^{n-k}b^k$, per quel indice specifico di k, che è ottenibile anche dal triangolo di tartaglia
		\item Triangolo di tartaglia:
		\[
		\begin{array}{ccccccccccc}
			&&&&1&&&\\
			&&&1&&1&&\\
			&&1&&2&&1&\\
			&1&&3&&3&&1\\
			1&&4&&6&&4&&1&
		\end{array}
		\]
		e così via, costruito dalla somma dei due elementi alla righe precedente, in cui partendo dall'alto dalla riga 0, fino alla riga n si ha da sinistra a destra l'indice k della riga n.
	\end{itemize}
	
	\subsection{Numeri interi}
	I numeri interi sono definiti come:
	\[\mathbb{Z} := \{..., -2, -1, 0, 1, 2, ...\}\]
	è possibile osservare che:
	\begin{itemize}
		\item $\mathbb{N} \subset \mathbb{Z}$
		\item $\mathbb{Z}$ non da un minimo, al contrario di $\mathbb{N}$
		\item $\forall n \in \mathbb{N}, n \not = 0, \exists! -n \in \mathbb{Z} \tc n + (-n) = 0$
	\end{itemize}
	Quindi $(\mathbb{Z}, +)$, ovvero l'insieme dei numeri interi in cui è definita la somma, è detto un gruppo commutativo/abeliano.
	Si dice abeliano, quando:
	\[\forall z_1,z_2 \in \mathbb{Z}, \quad z_1 + z_2 = z_2 + z_1\]
	Invece un insieme si definisce gruppo quando rispetta queste 3 proprietà:
	\[
	\left\{\begin{aligned}
		1 & \text{ la somma è associativa }: (a+b) + c = a + (b + c), \forall a,b,c \in \mathbb{Z}\\
		2 & \; \, \exists! 0, \text{ detto elemento neutro della somma } \tc 0 + z = z + 0 = z, \forall z \in \mathbb{Z}\\
		3 & \; \, \forall z \in \mathbb{Z} \exists! -z \in \mathbb{Z}, \text{ detto opposto di z } \tc z + (-z) = - z + z = 0
	\end{aligned}\right.\]
	
	\subsection{Numeri razionali}
	i numeri razionali sono definiti come:
	\[\mathbb{Q} := \{\frac{p}{q} : p \in \mathbb{Z}, q \in \mathbb{N}^+\}\]
	Allora $(\mathbb{Q}, +, \cdot)$, è un campo:
	\begin{itemize}
		\item $(\mathbb{Q}, +)$ è un gruppo commutativo.
		\item $(\mathbb{Q}, \cdot)$ è un gruppo commutativo.
		\item $\forall a,b,c \in \mathbb{Q}, a(b+c) = ab + ac$, quindi è valida la proprietà distributiva rispetto alla somma.
		\item Esistenza dell'elemento neutro del prodotto:
		\[a \cdot 1 = 1 \cdot a = a\]
		\item elemento inverso del prodotto:
		\[a^{-1} = \frac{1}{a}: a \cdot a^{-1} = a^{-1} \cdot a = \frac{a}{a} = 1\]
	\end{itemize}
	Altre proprietà di $\mathbb{Q}$, dette di ordinamento:
	\begin{itemize}
		\item Ordinamento totale: 
		\[\forall a,b \in \mathbb{Q}, a \leq b \wedge b \leq a\]
		\item Riflessiva:
		\[\forall a \in \mathbb{Q}, a \leq a\]
		\item Antisimmetrica:
		\[\forall a, b \in \mathbb{Q}, a \leq b \vee b\leq a \Rightarrow a = b\]
		\item Transitiva:
		\[\forall a,b,c \in \mathbb{Q}, a \leq b \wedge b \leq c \Rightarrow a \leq c\]
		\item $\forall a,b \in \mathbb{Q}, a \leq b \Rightarrow a+ c \leq b +c, \forall c \in \mathbb{Q}$
		\item $\forall a,b \in \mathbb{Q}, a \leq b \Rightarrow a \cdot c \leq b \cdot c, \forall c \in \mathbb{Q}_0^+$
	\end{itemize}
	
	Oss: $\mathbb{Q}$ ha dei buchi, infatti:
	\[\exists q : q^2 = 2 \Rightarrow q = \pm \sqrt{2} \not \in \mathbb{Q}\]
	
	\subsubsection{Dimostrazione irrazionalità di $\sqrt{2}$}
	\begin{proposition}
		$\not \exists q \in \mathbb{Q} \tc q^2 = 2 \rightarrow (\frac{p}{q})^2 = 2$, con $p$ e $q$ coprimi, e $q \not = 0$
	\end{proposition}
	
	\begin{proof}
		Dimostro per assurdo
		\[\left(\frac{p}{q}\right)^2 = 2 \Rightarrow p^2 = 2q^2\] 
		dato che $p^2$ è uguale a $2q^2$, p è pari, perciò può essere scritto come $(2k)^2$, svolgendo i calcoli.
		\[4k^2 = 2q^2 \Rightarrow 2k^2 = q^2\]
		ora anche $q^2$ è uguale a $2k^2$, quindi anche q è pari, perciò la nostra tesi non è più valida in quanto non è vero che p e q sono coprimi.
	\end{proof}
	
	\subsubsection{Scrittura di $q \in \mathbb{Q}$ forme decimali}
	$\mathbb{Q}$ corrisponde all'insieme $\{n,n_1n_2n_3...\}$, cioè $n \in \mathbb{Z}, n_i \in \{0, 1, 2, 3, ..., 9\} \tc $ sono finite $(n_i = 0, \forall i > i_0)$ o periodiche (un gruppo di n cifre si ripete all'infinito):
	\[\mathbb{Q} := \{n + \frac{n_1}{10} + \frac{n_2}{10^2} + ... + \frac{n_k}{10^k}\}, k \in \mathbb{N}\]
	
	\subsubsection{Caso interessante}
	Presi:
	\[A := \{q \in \mathbb{Q} \tc q^2 < 2\} \quad \quad B := \{q \in \mathbb{Q} \tc q^2 > 2\}\]
	
	\begin{center}
		\begin{tikzpicture}[scale=2]
			% Retta
			\draw[->, blue, thick] (-3,0) -- (3,0);			
			
			% Parentesi per gli insiemi
			\node at (0.1,0) {$(\;\;\;$}; % parentesi sinistra (A)
			\node at (-0.1,0) {$\;\;\;)$};  % parentesi destra (B)
			
			% Segmenti che indicano A e B (opzionale, più chiaro)
			\draw[thick, red] (-3,0) -- (0,0); % A
			\draw[thick, blue] (0,0) -- (3,0); % B
			
			% Punto radice di 2
			    \draw (0,0.1) -- (0,-0.1) node[below] {$\sqrt{2} = 1.4142135623...$};
		\end{tikzpicture}
	\end{center}
	
		
	osserviamo che:
	\begin{itemize}
		\item $a \in A, b \in B \Rightarrow a \leq b$
		\item A, B sono vicino quanto vogliamo:
		\[\exists a \in A, b \in B \tc a,b \text{ siano vicini}\]
		\item $\not \exists c \in \mathbb{Q} \tc a \leq c \leq b, \forall a \in A, \forall b \in B$, quindi $\mathbb{Q}$ non soddisfa l'assioma di Dedekind o assioma della completezza.
	\end{itemize}
	
	\subsection{Numeri reali}
	$(R, +, \cdot)$ è definito assiomaticamente tramite:
	\begin{itemize}
		\item $\mathbb{Q} \subset \mathbb{R}$
		\item $(R, +, \cdot)$ soddisfa le proprietà di $\mathbb{Q}$
		\item soddisfa l'assioma di Dedekind, quindi $\mathbb{R}$ non ha buchi, ovvero:
		\[\forall A, B \subset \mathbb{R} \tc A \not = \emptyset, B \not = \emptyset, A \cup B = \mathbb{R} \wedge a \leq b, \forall a \in A, \forall b \in B \Rightarrow \exists! c \in \mathbb{R} \tc a \leq c \leq b\]
		con c che può appartenere ad A o a B, ma non ad entrambi
	\end{itemize}
	
	\begin{center}
		\begin{tikzpicture}[scale=2]
			% Retta
			\draw[->, blue, thick] (-3,0) -- (3,0);			
			
			% Parentesi per gli insiemi
			\node at (0,0) [font=\large] {$\,($}; % parentesi sinistra (A)
			\node at (0,0) [font=\large] {$]\,$};  % parentesi destra (B)
			
			% Segmenti che indicano A e B (opzionale, più chiaro)
			\draw[thick, red] (-3,0) -- (0,0); % A
			\draw[thick, blue] (0,0) -- (3,0); % B
			
			% Punto radice di 2
			\draw (0,0.1) -- (0,-0.1) node[below] {$c$};
		\end{tikzpicture}
	\end{center}
	Importante: se $A \cup B \not = \mathbb{R}$, allora non è detto che c sia unico:
	\begin{center}
		\begin{tikzpicture}[scale=2]
			% Retta
			\draw[->, black, thick] (-3,0) -- (3,0);			
			
			% Parentesi per gli insiemi
			\node at (0.5,0) [font=\large] {$\,($}; % parentesi sinistra (A)
			\node at (-0.5,0) [font=\large] {$]\,$};  % parentesi destra (B)
			
			% Segmenti che indicano A e B (opzionale, più chiaro)
			\draw[thick, red] (-3,0) -- (-0.5,0); % A
			\draw[thick, blue] (0.5,0) -- (3,0); % B
			
			% Punto c
			\draw (0,0.1) -- (0,-0.1) node[below] {$c$};
			\draw (0.1,0.1) -- (0.1,-0.1) node[below] {$c$};
			\draw (-0.1,0.1) -- (-0.1,-0.1) node[below] {$c$};
			\draw (-0.2,0.1) -- (-0.2,-0.1) node[below] {$c$};
			\draw (0.2,0.1) -- (0.2,-0.1) node[below] {$c$};
		\end{tikzpicture}
	\end{center}
	Se invece l'unione tra i due insiemi non è nulla o unica, quindi $A \cap B \not = \emptyset$
	
	\begin{center}
		\begin{tikzpicture}[scale=2]
			% Retta
			\draw[->, black, thick] (-3,0) -- (3,0);			
			
			% Parentesi per gli insiemi
			\node at (0.5,0) [font=\large] {$\,]$}; % parentesi sinistra (A)
			\node at (-0.5,0) [font=\large] {$(\,$};  % parentesi destra (B)
			
			% Segmenti che indicano A e B (opzionale, più chiaro)
			\draw[thick, red] (-3,0) -- (+0.5,0); % A
			\draw[thick, blue] (-0.5,0) -- (3,0); % B
			\draw[thick, violet] (-0.55,0) -- (0.55,0);
			
			% Punto c
			\draw (0,0.1) -- (0,-0.1) node[below] {$c$};
		\end{tikzpicture}
	\end{center}
	in questo caso $\exists b \in B \tc b < c$
	
	\subsubsection{Teorema della caratterizzazione di $\mathbb{R}$}	
	$\mathbb{R}$ è l'unico campo ordinato che può essere rappresentato con l'insieme di tutti i possibili decimali allineati:
	\[\mathbb{R} := \{m,d_1d_2d_3...d_j \tc m \in \mathbb{Z}, d_j \in \{0, 1, ..., 9\} \tc j \in \mathbb{N}^+\}\}
	\]
	\[\setminus \{m,d_1d_2d_3...d_n \tc m \in \mathbb{Z}, d_j \in \{0, ..., 9\} \tc \exists j_0 \in \mathbb{N}^+ \tc d_j = 9,  \forall j \geq j_0 \}\]
	devono quindi essere esclusi tutti gli allineamenti di decimali in cui dopo un certo indice $j$ si susseguo sono 9, in quanto è lo stesso numero del successivo susseguito da tutti 0.
	
	Si dice che $\mathbb{Q}$ è denso in $\mathbb{R}$, quanto $\forall x \in \mathbb{R}\setminus \mathbb{Q}$ può essere approssimato da numeri razionali.
	
	\subsection{Intervalli, semirette ed estremi}
	Definizione:
	\begin{itemize}
		\item sia $A \subset \mathbb{R}, A \not = \emptyset$, è detto che $M_A$ è un \textbf{maggiorante} di A, se $m \geq a, \forall a \in A$.
		\[\mathcal{M}_A := \{ M_A \tc M_A \geq a, \forall a \in A\}\]
		ovvero l'insieme dei maggioranti
		\item sia $A \subset \mathbb{R}, A \not = \emptyset$, è detto che $m_A$ è un \textbf{minorante} di A, se $m \leq a, \forall a \in A$.
		\[\textsl{m}_A := \{ m_A \tc m_A \leq a, \forall a \in A\}\]
		ovvero l'insieme dei minoranti
		\item sia $A \subset \mathbb{R}, A \not = \emptyset$, A è \textbf{limitato superiormente} se $\mathcal{M}_A \not = 0$, quindi A ha almeno un maggiorante.
		
		N non è limitato superiormente in quanto $\forall n \in \mathbb{N}, \exists m > n \tc m \in \mathbb{N}$
		\item sia $A \subset \mathbb{R}, A \not = \emptyset$, A è \textbf{limitato inferiormente} se $\textsl{m}_A \not = 0$, quindi A ha almeno un minorante
		
		\item sia $A \subset \mathbb{R}, A \not = \emptyset$, allora il numero $M \in \mathbb{R}$, si dice \textbf{massimo di A}, $\max A$, se $M$ è un maggiorante di A e se $M \in A$.
		
		\item sia $A \subset \mathbb{R}, A \not = \emptyset$, allora il numero $m \in \mathbb{R}$, si dice \textbf{minimo di A}, $\min A$, se $m$ è un minorante di A e se $m \in A$.		
		
		\item sia $A \subset \mathbb{R}, A \not = \emptyset$, allora il minimo dei maggioranti di A, si dice \textbf{estremo superiore} di $A$, $\sup A$ o supremum $A$, nel caso in cui l'estremo superiore non esista, il $\sup A = +\infty$
		
		\item sia $A \subset \mathbb{R}, A \not = \emptyset$, allora il massimo dei minoranti di A, si dice \textbf{estremo inferiore} di $A$, $\inf A$ o infumum $A$, nel caso in cui l'estremo inferiore non esista, il $\inf A = -\infty$
	\end{itemize}
	
	\begin{proposition}
		sia $A \subset \mathbb{R}, A \not = \emptyset$, allora $\exists \sup A$
	\end{proposition}
	
	\begin{proof}
		$A $ è limitato superiormente $\Rightarrow \mathcal{M}_A \not = \emptyset$, per definizione: $\forall a \in A,\forall b \in \mathcal{M}_A$ si ha $a \leq b$.
		
		L'assioma di Dedekind implica che: $\exists c \in \mathbb{R} \tc a \leq c \leq b \Rightarrow c = min \mathcal{M}_A = \sup A$
	\end{proof}
	
	Sia $A$ limitato superiormente, quindi $\exists \sup A \in \mathbb{R}$, allora il $\sup A$ è caratterizzato da:
	\[\sup A = \min \mathcal{M}_A\]
	\[
	\left\{
	\begin{aligned}
		\sup A &\in \mathcal{M}_A \\
		\forall \lambda < &\sup A \Rightarrow \lambda\in \mathcal{M}_A
	\end{aligned}
	\right. \iff \left\{
	\begin{aligned}
	\sup A &\in \mathcal{M}_A \\
	\forall \lambda < &\sup A, \exists a \in A \tc \lambda < a
	\end{aligned}\right.
	\]
	Osservazione:
	sia $A \subset \mathbb{R} \tc \sup A \in \mathbb{R}$:
	\begin{itemize}
		\item se $\sup A \in A \Rightarrow \sup A = \max A$
	\end{itemize}
	Osservazione: sia $A \subset \mathbb{R} \tc \inf A \in \mathbb{R}$:
	\begin{itemize}
	\item se $\inf A \in A \Rightarrow \inf A = \min A$
	\end{itemize}	
	Importante:
	sia $A \subset \mathbb{R}, A \not = \emptyset \Rightarrow \exists \sup A, \exists \inf A $, ma non è detto che esistano il $\max A, \min A$.
	
	Osservazione:
	ogni $S \subset \mathbb{Z}, S \not = \emptyset$, limitato superiormente ha un massimo, mentre se limitato inferiormente ha minimo.
	
	\subsection{Proprietà di Archimede}
	$\forall a, b \in \mathbb{R}, a > 0 \implies \exists n \in \mathbb{N}^+ \tc na > b$.
	
	Idea:
	\[A :=  \{na \tc n \in \mathbb{N}^+\}\]
	Supponendo che $\not \exists n \in \mathbb{N}^+ \tc na > b \Rightarrow b \geq an, \forall n \in \mathbb{N}^+ \Rightarrow b \text{ è un maggiorante } \mathcal{M}_A \not = \emptyset $ %TO DO
	
	Conseguenze:
	\begin{itemize}
		\item $\mathbb{N}, \mathbb{Z}, \mathbb{Q}, \mathbb{R}$, non sono limitati superiormente, in quanto tutti contengono il precedente.
		\item $\forall x > 0, \exists n \in \mathbb{N}^+ \tc \frac{1}{n} < x$, che come corollario ha che negando la disuguaglianza, posso dire che:
		$x \geq 0 \tc \forall a \in \mathbb{N}^+, x < \frac{1}{n} \Rightarrow x = 0$
	\end{itemize}
	
	\subsection{Funzione modulo}
	\[|x| := 
	\left\{
	\begin{aligned}
		x & \quad \text{ se } x \geq 0\\
		-x & \quad \text{ se } x < 0
	\end{aligned}
	\right. \equiv |x| = \max\{x, -x\} \in [0, +\infty)
	\]
	
	\subsection{Intervalli}
	$I \subset \mathbb{R}$, dice intervallo se $\forall x,y \in I, \exists z \in \mathbb{R} \tc x < z < y, z \in I$
	
	Un intervallo può essere descritto come uno di questi 4 tipi, avendo, $a \leq b, a,b \in \mathbb{R} \cup \{-\infty, +\infty\}$
	\begin{itemize}
		\item $[a, b] := \{x \in \mathbb{R} \tc a \leq x \leq b\}$
		\item $]a, b] := \{x \in \mathbb{R} \tc a < x \leq b\}$
		\item $[a, b[ := \{x \in \mathbb{R} \tc a \leq x < b\}$
		\item $]a, b[ := \{x \in \mathbb{R} \tc a < x < b\}$
	\end{itemize}
	se a o b sono uguale a più o meno infinito, allora l'estremo si dice aperto.
	
	Definizione:
	$A \subset \mathbb{R}$ si dice denso in $\mathbb{R}$ se $\forall I \subset \mathbb{R}, \exists a \in A \tc a \in I$
	
	\section{Cenni su $\mathbb{R}^n$, cardinalità e numeri complessi}
	Definizione:
	\begin{itemize}
		\item $\mathbb{R} \times \mathbb{R} \equiv \mathbb{R}^2 = \{(x, y) \tc x, y \in \mathbb{R}\} \equiv$ piano cartesiano
		\item $\mathbb{R} \times \mathbb{R} \times \mathbb{R} \equiv \mathbb{R}^3$
		\item $\mathbb{R} \times \mathbb{R} \times ... \times \mathbb{R} \equiv \mathbb{R}^n$, e le sue coordinate sono: $(x_1, x_2,x_3, ..., x_ n) \in \mathbb{R}^n$
	\end{itemize}
	
	\subsection{Intorno di un punto}
	Per intorni si intende:
	\[J_{x_0} = J(x_0) = \{]a,b[ \tc x \in ]a,b[\}\]
	è definito come intorno particolare:
	\[Ir_0(x_0) = (x_0 - r_0, x_0 + r_0) = \{x \in \mathbb{R} \tc |x -x_0| < r_ 0\}\]
	un intervallo di centro $x_0$ e raggio $r_0$.
	
	$|\quad|$ permette di definire una distanza su $\mathbb{R}$, detta distanza euclidea:
	\[d(x, y) = |x - y|\]
	
	Osservazione:
	$\forall I \in J(x_0), \exists r > 0 \tc Ir(x_0) \subset J(x_0)$
	
	Prendendo: $r < \min\{d(x_0, a), d(x_0, b)\}$
	
	\begin{center}
		\begin{tikzpicture}[scale=2]
			% Retta
			\draw[->, blue, thick] (-3,0) -- (3,0);			
			
			% Parentesi per gli insiemi
			\node at (-0.6,0) [font=\large] {$\,($}; % parentesi sinistra (A)
			\node at (1.6,0) [font=\large] {$)\,$};  % parentesi destra (B)
						
			\node at (-0.3,0) [font=\large] {$\,($}; % parentesi sinistra (A)
			\node at (0.3,0) [font=\large] {$)\,$};  % parentesi destra (B)
			
			% Punto radice di 2
			\draw (0,0.1) -- (0,-0.1) node[below] {$X_0$};
		\end{tikzpicture}
	\end{center}
	
	\subsubsection{Distanza in $\mathbb{R}^n$}
	In $\mathbb{R}^2$:
	
	\begin{center}
		\begin{tikzpicture}[scale=1]
			% Assi cartesiani
			\draw[->] (-1,0) -- (5,0) node[right] {$x$};
			\draw[->] (0,-1) -- (0,5) node[above] {$y$};
			
			% Punto P1
			\filldraw[blue] (2,3) circle (2pt) node[above right] {$P_1(x_1,y_1)$};
			
			% Punto P2
			\filldraw[red] (1,1) circle (2pt) node[below left] {$P_2(x_2,y_2)$};
			
			\draw[thick, dashed] (2,3) -- (1,1) node[midway, above left] {$d$};;
		\end{tikzpicture}
	\end{center}
	
	$d(P_1, P_2) := \sqrt{(x_2 - x_1)^2 + (y_ 2 - y_2)^2} = |(x_2 - x_1) + (y_ 2 - y_2)| = ||P_2 - P _1||$.
	
	Nel caso in cui $x_1 = x_2$, allora è come si ci trovassimo in $\mathbb{R}$, quindi si può usare la formula per la distanza euclidea. \newline
	In $\mathbb{R}^3$:
	
	Presi due punti $P_1, P_2$:
	\[P_1 = (x^1_1, x^1_2, x^1_3, x^1_4, ..., x^1_n), \in \mathbb{R}^n\]
	\[P_1 = (x^2_1, x^2_2, x^2_3, x^2_4, ..., x^2_n), \in \mathbb{R}^n\]
	
	\[d(P_1, P_1) := ||P_2 - P_1|| = \sqrt{(x^1_1 - x^2_1)^2 + (x^1_2 - x^2_2)^2 + ... + (x^1_n - x^2_n)^2}\]
	
	\subsubsection{Intorno in $\mathbb{R}^n$}
	se $x_0 \in \mathbb{R}^2 \tc I_r(x_0) := \{(x_1,x_2) \in \mathbb{R}^2 \tc ||x - x_0|| < r\}$
	
	\begin{center}
		\begin{tikzpicture}[scale=1]
			% Assi cartesiani
			\draw[->] (-1,0) -- (5,0) node[right] {$x$};
			\draw[->] (0,-1) -- (0,5) node[above] {$y$};
			
			% Punto P1
			\filldraw[blue] (2,2) circle (2pt) node[above] {$P_1(x_1,y_1)$};
			
			\draw[thick, blue] (2,2) circle (1.7);
			
			\draw[thick, dashed] (2,2) -- (3.7,2) node[midway, below] {$r$};;
		\end{tikzpicture}
	\end{center}
	
	\subsection{Coordinate polari}
	preso $P = (x_1, y_1) \in \mathbb{R}^2$, è possibile trovare $\theta, \rho \tc:$
	\[\left\{\begin{aligned}
		x &= \rho \cos \theta\\
		y &= \rho \sin \theta
	\end{aligned}\right. \rightarrow \rho := d(0, P) = \sqrt{(x - 0)^2 + (y - 0)^2}\]
	
	\begin{center}
		\begin{tikzpicture}[->, thick]
			% assi
			\draw[->] (-3,0) -- (3,0) node[above] {$x$};
			\draw[->] (0,-3) -- (0,3) node[above] {$y$};
			
			% punti base
			\fill (1,0) circle (2pt) node[below] {$(1,0)$};
			\fill (0,1) circle (2pt) node[left] {$(0,1)$};
			
			% vettore z
			\draw[->, black, thick] (0,0) -- (2,0.8) node[above] {$P$};
			
			\draw[black] (0,0) -- (2,0.8) node[midway, above] {$\rho$};
			
			% nodi
			\coordinate (X) at (2,0);
			\coordinate (O) at (0,0);
			\coordinate (Z) at (2,0.8);
			
			% angolo senza freccia
			\pic [draw=black, "$\theta$", angle eccentricity=1.5, angle radius=1cm, mark=none]
			{angle = X--O--Z};
		\end{tikzpicture}
	\end{center}
	
	La somma tra le componenti al quadrato di un punto equivale a $\rho$ al quadrato, secondo il teorema di Pitagora.
	\[x^2+y^2 = \rho^2 (\cos^2\theta + \sin^2\theta) = \rho^2\]
	
	\subsection{Retta in $\mathbb{R}^2$}
	Una retta non verticale è definita da una coppia ordinata:
	\[r := \{(x,y) \in \mathbb{R}^2 \tc y = mx + q\}\]
	
	\begin{center}
		\begin{tikzpicture}[scale=1]
		% Assi solo nel primo quadrante
		\draw[->] (-1,0) -- (6,0) node[right] {$x$};
		\draw[->] (0,-1) -- (0,6) node[above] {$y$};
		
		% Punti P1 e P2 nel primo quadrante
		\coordinate (P1) at (1,2);
		\coordinate (P2) at (4,3);
		
		% Etichetta i punti
		\filldraw[blue] (P1) circle (2pt) node[below right] {$P_1(x_1,2_2)$};
		\filldraw[red] (P2) circle (2pt) node[above left] {$P_2(x_2,y_2)$};
		
		% Calcolo della retta passante per i due punti
		% m = (3 - 2)/(4 - 1) = 1/3 → y = (1/3)x + q
		% Inseriamo P1 per calcolare q:
		% 2 = (1/3)*1 + q → q = 2 - 1/3 = 5/3
		
		% Quindi il punto Q = (0, 5/3)
		\coordinate (Q) at (0,1.6667); % 5/3 ≈ 1.6667
		\filldraw[black] (Q) circle (2pt) node[left] {$q$};
		
		% Disegna la retta: y = (1/3)x + 5/3
		\draw[thick, dashed, domain=-1:6] plot (\x,{(1/3)*\x + 5/3});
		
	\end{tikzpicture}
	\end{center}
	
	in una retta $m$ è il coefficiente angolare della retta, definito dal rapporto incrementale:
	\[m := \frac{\Delta y}{\Delta x} = \frac{y_2 - y_1}{x_2 - x_1}\]
	e per q, si intende la quota del punto  di intersezione con l'asse delle y.
	
	Le rette verticali non sono descrivibili con questa equazione, in quanto si dividerebbe per 0 nel rapporto incrementale, perciò si usa una insieme numerico:
	\[r_v := \{(x,y) \in \mathbb{R}^2 \tc x = x_0\}\]
	
	\subsection{Numeri complessi ($\mathbb{C} \equiv \mathbb{R}^2$)}
	I numeri complessi sono definiti da una parte reale $a$ e da una parte immaginaria $b$, che moltiplica l'unità immaginaria, $i \in \mathbb{C} \tc i^2 = -1$.
	\[z \in \mathbb{C} \Rightarrow z = a + bi = (a, b), \text{ con } a,b \in \mathbb{R}\] 
	
	Preso un numero complesso $z = (a, b)$, è possibile trovare:
	\begin{itemize}
		\item l'opposto: $-z = (-a, -b)$
		\item il coniugato: $\overline{z} = (a, -b)$
		\item l'opposto del coniugato: $-\overline{z} = (-a, b)$
	\end{itemize}
	
	\begin{center}
		\begin{tikzpicture}[scale=0.8]
		% Assi cartesiani
		\draw[->] (-5,0) -- (5,0) node[right] {$\text{Re}$};
		\draw[->] (0,-5) -- (0,5) node[above] {$\text{Im}$};
		
		% Punti
		\coordinate (Z) at (2,3);            % z
		\coordinate (NegZ) at (-2,-3);       % -z
		\coordinate (ConjZ) at (2,-3);       % conjugate
		\coordinate (NegConjZ) at (-2,3);    % -conjugate
		
		% Disegna i punti
		\filldraw[blue] (Z) circle (2pt) node[above right] {$z = 2 + 3i$};
		\filldraw[red] (NegZ) circle (2pt) node[below left] {$-z = -2 - 3i$};
		\filldraw[green!70!black] (ConjZ) circle (2pt) node[below right] {$\overline{z} = 2 - 3i$};
		\filldraw[purple] (NegConjZ) circle (2pt) node[above left] {$-\overline{z} = -2 + 3i$};
		
		% Opzionale: linee tratteggiate verso gli assi
		\draw[dashed, gray] (Z) -- (2,0);
		\draw[dashed, gray] (Z) -- (0,3);
		
		\draw[dashed, gray] (NegZ) -- (-2,0);
		\draw[dashed, gray] (NegZ) -- (0,-3);
		
		\draw[dashed, gray] (ConjZ) -- (2,0);
		\draw[dashed, gray] (ConjZ) -- (0,-3);
		
		\draw[dashed, gray] (NegConjZ) -- (-2,0);
		\draw[dashed, gray] (NegConjZ) -- (0,3);
		\end{tikzpicture}
	\end{center}
		
	\subsection{Insiemi aperti e chiusi}
	Un insieme si dice aperto:
	\[A \subset \mathbb{R}^n \text{ si dice \textbf{aperto} se } \forall x \in A, \exists r>0 \tc Ir(x) \subset A\]
	Un insieme si dice chiuso:
	\[A \subset \mathbb{R}^n \text{ si dice \textbf{chiuso} se } A^c = \mathbb{R}^n \setminus A \text{ è aperto}\]
	Un insieme può essere ne aperto ne chiuso , ad esempio $(a, b]$ \newline
	Per punto interno si intende:
	\[A \subset \mathbb{R}^n, x \in \mathbb{R}^n \text{ si dice \textbf{punto interno} ad A se } \exists r>0 \tc Ir(x) \subset A\]
	l'insieme di tutti i punti interni è definito da:
	\[\mathring{A} := \{x \in \mathbb{R}^n \tc x \text{ è interno ad A}\}\]
	Osservazione: 
	\[\left\{
	\begin{aligned}
	\text{ se } & A \text{ è aperto } \mathring{A} = A \\
	\mathring{A} &\text{ è il più grande insieme aperto } B \tc B \subset A \end{aligned}
	\right\} \Rightarrow A \text{ è aperto } \iff \mathring{A} = A\]
	Per punto esterno si intende:
	\[A \subset \mathbb{R}^n, x \in \mathbb{R}^n, \text{ è esterno ad A se } \exists r>0 \tc Ir(x) \cap A \not = \emptyset\]
	oppure dicendo che è interno all'insieme complementare di A:
	\[Ir(x_0) \subset A^c \iff x \text{ è interno ad } A^c\]
	Per punto di bordo, o frontiera, si intende un punto che non è ne interno ne esterno, quindi che si trova sul bordo dell'insieme.
	\[\delta A := \{x \in \mathbb{R}^n \tc \text{x sia un punto di bordo di A}\}\]
	
	Osservazioni:
	\begin{itemize}
		\item $\delta A = \delta A ^c$
		\item $\overline{A} := A \cup \delta A \equiv \text{ chiusura di A, è sempre un insieme chiuso}$
		\item $\overline{A} \text{ è il più piccolo insieme B chiuso} \tc A \subset B$
	\end{itemize}
	
	Esempio:
	
	\begin{center}
		\begin{tikzpicture}[scale=2]
			% Retta
			\draw[->, black, thick] (-3,0) -- (3,0);			
			
			% Parentesi per gli insiemi
			\node at (0.5,0) [font=\large] {$\,]$}; % parentesi sinistra (A)
			\node at (-1,0) [font=\large] {$(\,$};  % parentesi destra (B)
						
			% Punto c
			\draw (-1,-0.1) -- (-1,-0.1) node[below] {$0$};
			\draw (0.5,-0.1) -- (0.5, -0.1) node[below] {$1$};
		\end{tikzpicture}
	\end{center}
	
	\[A = (0, 1], \mathring{A} = (0,1), \delta A = \{0,1\}, \overline{A} = A \cup \delta A = [0, 1]\]
	Per punto isolato si intende:
	\[A \subset \mathbb{R}^n, x \in \mathbb{R}^n \text{ si dice isolato se } \exists r > 0 \tc Ir(x) \cap A = \{x\}\]
	Esempio:
	\[A = \{0\} \cup (3,4)\]
	
	\begin{center}
		\begin{tikzpicture}[scale=2]
			% Retta
			\draw[->, black, thick] (-3,0) -- (3,0);			
			
			% Parentesi per gli insiemi
			\node at (1,0) [font=\large] {$\,]$}; % parentesi sinistra (A)
			\node at (-0.5,0) [font=\large] {$(\,$};  % parentesi destra (B)
			\node at (-1.5, 0) [font=\large] {$|\,$};
			
			% Punto c
			\draw (-0.5,-0.1) -- (-0.5,-0.1) node[below] {$3$};
			\draw (1,-0.1) -- (1, -0.1) node[below] {$4$};
			\draw (-1.5,-0.1) -- (-1.5, -0.1) node[below] {$0$};
		\end{tikzpicture}
	\end{center}
	
	 $\Rightarrow p = 0 \text{ è punto isolato di } A$
	 
	 
	 in $\mathbb{N}$ ogni punto $n \in \mathbb{N}$ è isolato. \newline	 
	 Un insieme tale che ogni punto è isolato si dice ISOLATO.
	 \[A \subset \mathbb{R}^n, x \in \mathbb{R}^n \text{ si dice punto di accumulazione di A se } \forall Ir(x), (Ir(x) \setminus \{x\}) \cap A \not = \emptyset\]
	 \[A = (1,3)\setminus \{2\}\]
	 \begin{center}
	 	\begin{tikzpicture}[scale=2]
	 		% Retta
	 		\draw[->, black, thick] (-3,0) -- (3,0);			
	 		
	 		% Parentesi per gli insiemi
	 		\node at (0.5,0) [font=\large] {$\,)$}; % parentesi sinistra (A)
	 		\node at (-1,0) [font=\large] {$(\,$};  % parentesi destra (B)
	 		
	 		% Punto c
	 		\draw (-1,-0.1) -- (-1,-0.1) node[below] {$1$};
	 		\draw (0.5,-0.1) -- (0.5, -0.1) node[below] {$3$};
	 		
	 		\coordinate (Acc) at (-0.25,0);    % -conjugate
	 		
	 		% Disegna i punti
	 		\draw[blue] (Acc) circle (2pt) node[below] {$2$};
	 	\end{tikzpicture}
	 \end{center}
	 
	 Allora $x = 2$ è un punto di accumulazione, e non appartiene ad A, ma potrebbe anche appartenere.
	 
	 In $\mathbb{R}$:
	 \[\overline{\mathbb{R}} := \mathbb{R} \cup \{\pm \infty\}\]
	 Allora un punto si dice di accumulazione, se $\forall u \subset J(x), (u \setminus \{x\}) \cap A \not = \emptyset$
	 
	 osservazione:
	 \begin{itemize}
	 	\item se $x \in \mathbb{R} \Rightarrow u = Ir(x)$, come per la definizione precedente
	 	\item $x = + \infty$, $+ \infty$ è un punto accumulazione di A, se $\forall u \in J(+\infty)$, si ha che $(u \setminus \{+\infty\}) \cap A \not = 0 \rightarrow u = (M, +\infty)$, quindi A non è limitato superiormente.
	 	\item $x = - \infty$, $- \infty$ è un punto accumulazione di A, se $\forall u \in J(-\infty)$, si ha che $(u \setminus \{-\infty\}) \cap A \not = 0 \rightarrow u = (-\infty, m)$, quindi A non è limitato inferiormente.
	 \end{itemize}
	 Caratterizzazione di $\overline{A} = A \cup \delta A$:
	 \[\overline{A} = \{x \in \mathbb{R}^n \tc \forall r > 0, (Ir(x) \cap A) \not = 0\}, \text{ l'insieme dei punti aderenti ad A}\]
	 
	 Osservazione:
	 \[A \subset \mathbb{R}^n \Rightarrow \overline{A} \subset \overline{\mathbb{R}} = A \subset \mathbb{R} \cup \{\pm \infty\}\]
	 \[+ \infty \in \overline{A} \iff \text{ $+\infty$ è un punto di accumulazione per $A$}\]
	 \[\overline{A} := \{x \in \overline{\mathbb{R}} \tc \forall U \subset J(x) \tc \}\]
	 
	 Esercitazione:
	 \[A = \mathbb{Q} \cap [0,1]\]
	 
	 \begin{center}
	 	\begin{tikzpicture}[scale=2]
	 		% Retta
	 		\draw[->, black, thick] (-3,0) -- (3,0);			
	 		
	 		% Parentesi per gli insiemi
	 		\node at (0.5,0) [font=\large] {$\,]$}; % parentesi sinistra (A)
	 		\node at (-1,0) [font=\large] {$[\,$};  % parentesi destra (B)
	 		
	 		% Punto c
	 		\draw (-1,-0.1) -- (-1,-0.1) node[below] {$0$};
	 		\draw (0.5,-0.1) -- (0.5, -0.1) node[below] {$1$};
	 	\end{tikzpicture}
	 \end{center}
	 
	 \[\mathring{A} = \overline{\mathbb{Q} \cap [0,1]} = \emptyset\]
	 
	 \[\overline{A} = \overline{\mathbb{Q} \cap [0,1]} = [0,1]\]
	 \[x \in \overline{\mathbb{Q} \cap [0,1]} \iff \exists r > 0 \tc Ir(x) \cap (\mathbb{Q} \setminus [0,1]) \not = 0, \text{ Q è denso in R}\]
	 
	 \[\delta A = \delta (\mathbb{Q} \cap [0,1]) = [0,1] \rightarrow \delta A = \overline{A} \setminus \mathring{A} = [0,1] \setminus \emptyset\]
	 
	 Dimostrazione che la cardinalità di $\mathbb{N} = \mathbb{Q} < \mathbb{R}$
	 
	 
\end{document}
