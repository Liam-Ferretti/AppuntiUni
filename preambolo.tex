% -------------------------------
% Pacchetti di base
% -------------------------------
\usepackage[utf8]{inputenc}
\usepackage[T1]{fontenc}
\usepackage[italian]{babel}
\usepackage{amsmath, amssymb, amsthm, mathtools, physics}
\usepackage{graphicx}
\usepackage{tikz}
\usetikzlibrary{calc}
\usetikzlibrary{intersections}
\usepackage{hyperref}
\usepackage{geometry}
\newtheorem{proposition}{Proposizione}
\geometry{a4paper, margin=2.5cm}
\newcommand{\tc}{\;\text{t.c.}\;}

\usetikzlibrary{arrows.meta}
\usetikzlibrary{angles,quotes, arrows.meta}

\usepackage{verbatim}


% -------------------------------
% Per codice (Lab Calcolo)
% -------------------------------
\usepackage{listings}
\usepackage{xcolor}

\lstnewenvironment{code}
{\lstset{
		language=C++,
		inputencoding=utf8,
		extendedchars=true,
		literate={à}{{\`a}}1
		{è}{{\`e}}1
		{é}{{\'e}}1
		{ì}{{\`i}}1
		{ò}{{\`o}}1
		{ù}{{\`u}}1,
		basicstyle=\ttfamily\footnotesize,
		numbers=left,
		numberstyle=\tiny,
		frame=single,
		breaklines=true,
		showstringspaces=false,
		commentstyle=\color{gray},  % commenti verdi
		keywordstyle=\color{cyan},      % parole chiave azzurre
		stringstyle=\color{orange}      % stringhe arancioni
}}
{}




