% -------------------------------
% Pacchetti di base
% -------------------------------
\usepackage[utf8]{inputenc}
\usepackage[T1]{fontenc}
\usepackage[italian]{babel}
\usepackage{amsmath, amssymb, amsthm, mathtools, physics}

\usepackage{tcolorbox}
\tcbuselibrary{breakable} % per box su più righe

% Comando personalizzato per teoremi/enunciati
\newtcolorbox{teorembox}{
	colback=gray!10,    % sfondo grigio chiaro
	colframe=gray!50,   % bordo grigio
	boxrule=0.5pt,      % spessore bordo
	sharp corners,      % angoli non arrotondati
	breakable,          % consente il ritorno a capo nel box
	left=8pt, right=8pt, top=6pt, bottom=6pt,
}

% Comando semplificato: \teorem{...}
\newcommand{\teorem}[1]{\begin{teorembox}#1\end{teorembox}}
\newcommand{\brackets}[1]{\langle #1 \rangle}
\usepackage{graphicx}
\usepackage{tikz}
\usetikzlibrary{calc}
\usetikzlibrary{intersections}
\usepackage{pgfplots}
\pgfplotsset{compat=1.18}
\usepackage{hyperref}
\usepackage{geometry}
\newtheorem{proposition}{Proposizione}
\geometry{a4paper, margin=2.5cm}
\newcommand{\tc}{\;\text{t.c.}\;}

\usetikzlibrary{arrows.meta}
\usetikzlibrary{angles,quotes, arrows.meta}

\usepackage{verbatim}


% -------------------------------
% Per codice (Lab Calcolo)
% -------------------------------
\usepackage{listings}
\usepackage{xcolor}

\lstnewenvironment{code}
{\lstset{
		language=C,
		inputencoding=utf8,
		extendedchars=true,
		literate={à}{{\`a}}1
		{è}{{\`e}}1
		{é}{{\'e}}1
		{ì}{{\`i}}1
		{ò}{{\`o}}1
		{ù}{{\`u}}1,
		basicstyle=\ttfamily\footnotesize,
		numbers=left,
		numberstyle=\tiny,
		frame=single,
		breaklines=true,
		showstringspaces=false,
		commentstyle=\color{gray},  % commenti verdi
		keywordstyle=\color{cyan},      % parole chiave azzurre
		stringstyle=\color{orange}      % stringhe arancioni
}}
{}

\usepackage{lmodern}
\usepackage{inconsolata} % font monospaziato
\usepackage{xcolor}
\usepackage{listings}

% Stile UNIX
\lstdefinestyle{unixstyle}{
	language=bash,
	basicstyle=\ttfamily\small,
	numbers=left,                     % ✅ numeri di riga
	numberstyle=\tiny\color{gray!70}, % colore numeri
	numbersep=6pt,
	frame=single,
	backgroundcolor=\color{gray!6},
	rulecolor=\color{black!60},
	columns=fullflexible,
	keepspaces=false,                 % ✅ rimuove tab/spazi iniziali
	showstringspaces=false,
	breaklines=true,
	aboveskip=0pt,
	belowskip=0pt,
	xleftmargin=0pt,
	xrightmargin=0pt,
	upquote=true,
	literate={\$}{{\textcolor{black}{\$}}}1
}

% Definizione del comando \unix{...}
\newcommand{\unix}[1]{%
	\lstinline[style=unixstyle]!#1!%
}



