\documentclass[a4paper,12pt]{article}

% pacchetti comuni
% -------------------------------
% Pacchetti di base
% -------------------------------
\usepackage[utf8]{inputenc}
\usepackage[T1]{fontenc}
\usepackage[italian]{babel}
\usepackage{amsmath, amssymb, amsthm, mathtools, physics}
\usepackage{graphicx}
\usepackage{tikz}
\usepackage{hyperref}
\usepackage{geometry}
\geometry{a4paper, margin=2.5cm}

% -------------------------------
% Per codice (Lab Calcolo)
% -------------------------------
\usepackage{listings}
\usepackage{xcolor}

\lstnewenvironment{code}
{\lstset{
		language=C++,
		inputencoding=utf8,
		extendedchars=true,
		literate={à}{{\`a}}1
		{è}{{\`e}}1
		{é}{{\'e}}1
		{ì}{{\`i}}1
		{ò}{{\`o}}1
		{ù}{{\`u}}1,
		basicstyle=\ttfamily\footnotesize,
		numbers=left,
		numberstyle=\tiny,
		frame=single,
		breaklines=true,
		showstringspaces=false,
		commentstyle=\color{gray},  % commenti verdi
		keywordstyle=\color{cyan},      % parole chiave azzurre
		stringstyle=\color{orange}      % stringhe arancioni
}}
{}






\usepackage{listings}
\lstset{
	inputencoding=utf8,        % interpretare il codice in UTF-8
	extendedchars=true,        % permette caratteri accentati nei commenti
	language=Python,           % linguaggio di default
	basicstyle=\ttfamily\footnotesize,
	breaklines=true,
	frame=single,
	numbers=left,              % numeri di riga a sinistra
	numberstyle=\tiny,         % dimensione numeri di riga
	commentstyle=\color{blue}, % colore commenti
	keywordstyle=\color{red}   % colore keyword
}

% info documento
\title{Appunti di Laboratorio di Calcolo}
\author{Liam Ferretti}
\date{\today}

\begin{document}
	
	\maketitle
	
	\begin{abstract}
		Affinché si possa svolgere la prova di esonero bisogna aver non più di 3 assenze in laboratorio.
		È necessario installare sul proprio computer un compilatore C gcc, e un interprete Python 3.x.
		Per i ricevimenti con Boeri si possono effettuare il mercoledì dalle 11 alle 12 nella stanza 407 al quarto piano dell'edificio "Fermi", per Spagnolo il martedì dalle 12 alle 13 nella stanza 116 al primo piano dell'edificio "Fermi".
		Si raccomanda l'acquisto del libro di testo: "Programmare la scienza L.Barone", che si potrà portare all'esame.
		Il laboratorio Pontecorvo si trova in via Tiburtina 205.
		
		Esame sarà una prova pratica di 3 ore, ma si potranno avere punti bonus ottenuti per chi frequenta il laboratorio all'esonero, il punteggio bonus è definito dal voto dell'esonero:
		\begin{itemize}
			\item 18-23: 1 punti
			\item 24-25: 2 punti
			\item 26-27: 3 punti
			\item 28-29: 4 punti
			\item 30/30 e lode: si è esonerati dalla prova finale.
		\end{itemize}
		Gli argomenti del corso saranno:
		\begin{itemize}
			\item Parte 1: Come funziona un computer
			\item Parte 2: Elementi di base del C
			\item Parte 3: Elementi avanzati del C
			\item Parte 4: Applicazioni e algoritmi + Python
		\end{itemize}
	\end{abstract}
	
	\newpage
	\tableofcontents
	\clearpage
	
	\section{Principi di funzionamento di un computer}
	Esistono diversi \textbf{sistemi di numerazione}, ovvero delle composizioni di regole con cui vengono rappresentati i numeri, i computer usano il binario, le cui cifre sono: $\{0, 1\}$
	Nella base decimale, quella che si usa quotidianamente, le cifre sono: $\{0, 1, ..., 8, 9\}$.
	Sono sistemi posizionali in cui, quindi, la posizione delle cifre.
	
	Es: $(123)_{10} = 1*10^2 + 2*10^1 + 3*10^0$
	
	$(101)_2 = 1*2^2+0*2^1+1*2^0$
	
	\subsection{Conversione decimale -> binario}
	Per convertire un numero da decimale a binario, si usa il metodo dei resti, ovvero si prende il numero, lo si divide per la base, e si scrive il resto, e si ripete, quando il numero sarà minore della base si trascrive quel numero come resto, e poi si scrive il numero in binario prendendo i resti dal basso verso l'alto.
	
	Es: n = $(43)_{10} \rightarrow (101011)_2$
	\[
	\begin{array}{c | c}
		43 & 2 : 1\\
		21 & 2 : 1\\
		10 & 2 : 0\\
		5 & 2 : 1\\
		2 & 2 : 0\\
		1 & 2 : 1\\
		0 & 2 \; \; \; \; \, \\
	\end{array}
	\]
		
\end{document}
